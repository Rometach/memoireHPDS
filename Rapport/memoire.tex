\documentclass[a4paper,12pt]{article}


%\usepackage[boxruled,vlined,english]{algorithm2e}
%\usepackage[francais,english]{babel}
%\usepackage[utf8]{inputenc}
%\usepackage[T1]{fontenc}
%\usepackage{hyperref}
%\usepackage{graphicx}
%\usepackage{lmodern}
%\usepackage{latexsym}
%\usepackage{csquotes}
%\usepackage{setspace}
%\usepackage{amsmath}
%\usepackage{amssymb}
%\usepackage{amsthm}
%\usepackage{amscd}
%\usepackage{color}
%\usepackage{calc}
%\usepackage[notes,backend=biber]{biblatex-chicago}
%\bibliography{biblio/mabiblio.bib}



\usepackage{helvet}
\renewcommand{\familydefault}{\sfdefault}

%\SetKw{Edb}{Side effect}
%\SetKw{Et}{and}
%\SetKw{Ou}{or}
%\SetKw{De}{from}
%\SetKw{A}{to}
%\SetKw{Par}{by}
%\SetKwBlock{Debut}{Begin}{End}
%\SetKwIF{Si}{SinonSi}{Sinon}{If}{then}{Else if}{Else}{EndIf}
%\SetKwFor{Pour}{For}{do}{EndFor}
%\SetKwFor{PourTout}{For all}{do}{EndFor}
%\SetKwFor{TantQue}{While}{do}{EndWhile}
%\SetKw{Retour}{Return}

\newcommand{\guill}[1]{“#1”}

\newcommand{\eme}[0]{$^\text{e}$}

\newcommand{\bigO}[1]{\mathcal O\left( #1 \right)}
\newcommand{\bigOmega}[1]{\Omega\left( #1 \right)}
\newcommand{\bigTheta}[1]{\Theta\left( #1 \right)}

\usepackage{helvet} % OBLIGATOIRE
\renewcommand{\familydefault}{\sfdefault} % OBLIGATOIRE





\setlength{\voffset}{-3.75cm}
\setlength{\hoffset}{-2.6cm}
\setlength{\oddsidemargin}{2.5cm} % OBLIGATOIRE
\setlength{\evensidemargin}{2.5cm} % OBLIGATOIRE
\setlength{\topmargin}{3.5cm} % OBLIGATOIRE !!!
\setlength{\headheight}{0in}
\setlength{\headsep}{0in}
\setlength{\topskip}{0in}
\setlength{\parindent}{0cm}
\setlength{\parskip}{1ex plus0.4ex minus0.2ex}
\setlength{\textwidth}{16cm} % OBLIGATOIRE
\setlength{\textheight}{24.7cm} % OBLIGATOIRE
\renewcommand{\baselinestretch}{1.5} % OBLIGATOIRE
\flushbottom
\setcounter{page}{1}
\setcounter{tocdepth}{2}



\title{\Large Internship report \\ \LARGE Computational analysis of jazz chord sequences}
\author{\normalsize Romain \textsc{Versaevel}, M1 Informatique Fondamentale, ENS de Lyon\\
\normalsize Tutored by David \textsc{Meredith}, Associate Professor, Aalborg University,\\
\normalsize leader of the Music Informatics and Cognition group\\}
\date{\today}

\begin{document}

\maketitle
\newpage

Le mémoire doit contenir essentiellement de travaux de recherche originaux et ne doit être un document de synthèse relatant des travaux d’autres chercheurs. Sous une forme ou une autre, doivent donc être dégagés: \\
* l’état de la recherche sur le sujet avant le début de celle-ci, \\
* La thèse (l’argument positif central que vous défendez dans votre mémoire et lors
de votre soutenance). \\
* la problématique et les objectifs, \\
* une réflexion sur ce qui a été fait et ce qui reste à faire. \\
D’une manière générale et dans tout le mémoire, il faut distinguer clairement ce qui est rappelé pour la clarté de l’exposition, de ce qui est un apport novateur. Toutes les citations doivent renvoyer à une référence précise (y compris la page), en étant lucide vis-à-vis des différents types de sources. De façon générale, toutes les affirmations doivent être étayées (comme si l’avocat de la partie adverse cherchait la faille dans votre enquête), qu’il s’agisse d’un mémoire de didactique ou d’histoire des sciences.

\newpage
\begin{abstract}
CECI EST UN RÉSUMÉ
\end{abstract}
\bigskip
\textbf{Mots-clefs :} 1 2 3 4 5 6

\newpage
\section{Bibliographie}
\printbibliography

\end{document}