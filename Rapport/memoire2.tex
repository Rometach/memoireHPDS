\documentclass[a4paper,12pt]{article}

% PACKAGES
\usepackage[T1]{fontenc}
\usepackage[utf8]{inputenc}
\usepackage[french]{babel}
\usepackage{csquotes}
\usepackage[notes,backend=biber]{biblatex-chicago}
\bibliography{biblio/mabiblio.bib}
\usepackage{amsmath}
\usepackage{amssymb}
\usepackage{amsthm}
\usepackage{amscd}
\usepackage{lmodern} % OBLIGATOIRE
\usepackage{textcomp}
\usepackage{graphicx}
\usepackage{hyperref}
\usepackage{sectsty}

% MISE EN PAGE
\setlength{\voffset}{-3.75cm}
\setlength{\hoffset}{-2.6cm}
\setlength{\oddsidemargin}{2.5cm} % OBLIGATOIRE
\setlength{\evensidemargin}{2.5cm} % OBLIGATOIRE
\setlength{\topmargin}{3.1cm} % OBLIGATOIRE
\setlength{\headheight}{0in}
\setlength{\headsep}{0in}
\setlength{\topskip}{0in}
\setlength{\parindent}{0cm}
\setlength{\parskip}{1ex plus0.4ex minus0.2ex}
\setlength{\textwidth}{16cm} % OBLIGATOIRE
\setlength{\textheight}{24.7cm} % OBLIGATOIRE
\renewcommand{\baselinestretch}{1.5} % OBLIGATOIRE
\flushbottom
\setcounter{page}{1}
\setcounter{tocdepth}{3}
\usepackage{helvet} % OBLIGATOIRE
\renewcommand{\familydefault}{\sfdefault} % OBLIGATOIRE

% PERSO
\newcommand{\guill}[1]{«~#1~»}
\newcommand{\guilldeux}[1]{“#1”}
\newcommand{\eme}[0]{$^\text{e}$}
\newcommand{\Ier}[0]{I$^\text{er}$} \newcommand{\IIe}[0]{II\eme} \newcommand{\IIIe}[0]{III\eme} \newcommand{\IVe}[0]{IV\eme} \newcommand{\Ve}[0]{V\eme} \newcommand{\VIe}[0]{VI\eme} \newcommand{\VIIe}[0]{VII\eme} \newcommand{\VIIIe}[0]{VIII\eme} \newcommand{\IXe}[0]{IX\eme} \newcommand{\Xe}[0]{X\eme} \newcommand{\XIe}[0]{XI\eme} \newcommand{\XIIe}[0]{XII\eme} \newcommand{\XIIIe}[0]{XIII\eme} \newcommand{\XIVe}[0]{XIV\eme} \newcommand{\XVe}[0]{XV\eme} \newcommand{\XVIe}[0]{XVI\eme} \newcommand{\XVIIe}[0]{XVII\eme} \newcommand{\XVIIIe}[0]{XVIII\eme} \newcommand{\XIXe}[0]{XIX\eme} \newcommand{\XXe}[0]{XX\eme} \newcommand{\XXIe}[0]{XXI\eme}
\newcommand{\bigO}[1]{\mathcal O\left( #1 \right)}
\newcommand{\bigOmega}[1]{\Omega\left( #1 \right)}
\newcommand{\bigTheta}[1]{\Theta\left( #1 \right)}
\newcommand{\zitat}[2]{\#Citation(#2)\#}
\newcommand{\maze}[0]{\emph{m\symbol{64}ze\textdegree2}}
\newcommand{\tpp}[0]{[\dots]}
\newcommand{\module}[1]{\texttt{\textsc{#1}}}
\newcommand{\patch}[1]{[\texttt{#1}]}








\title{\Large Internship report \\ \LARGE Computational analysis of jazz chord sequences}
\author{\normalsize Romain \textsc{Versaevel}, M1 Informatique Fondamentale, ENS de Lyon\\
\normalsize Tutored by David \textsc{Meredith}, Associate Professor, Aalborg University,\\
\normalsize leader of the Music Informatics and Cognition group\\}
\date{\today}

\begin{document}

%\maketitle
%\newpage

\tableofcontents

\newpage
\section{Introduction}

Cette annexe contient quelques ressources supplémentaires. Une première partie recense des liens, permettant entre autres d'accéder à des médias différents (programmes, enregistrements). La deuxième contient la liste des morceaux et programmes créés par Essl, annotés en fonction du rôle qu'y joue l'informatique. La troisième résume la partie biographique du document principal sous la forme d'une chronologie. La quatrième partie est la transcription d'un entretien réalisé avec Essl dans le cadre de la rédaction de ce document.

Cette annexe et le document principal sont disponibles au format PDF sur le dépôt Git suivant :
\href{https://github.com/Rometach/memoireHPDS}{https://github.com/Rometach/memoireHPDS}

\newpage
\section{Liens}

Cette partie propose quelques liens Internet (consultés le \today), qui fournissent des pistes d'approfondissement pour lecteur, donnant notamment accès aux programmes mentionnés et à des enregistrements de morceaux ou performances d'Essl. Dans la version PDF, les liens sont directement opérationnels.

\subsubsection*{Pages d'Essl}

\begin{itemize}
\item Sa page personnelle : \href{http://www.essl.at/}{http://www.essl.at/}
\item Sa chaîne vidéo sur YouTube : \href{https://www.youtube.com/c/KarlheinzEssl}{https://www.youtube.com/c/KarlheinzEssl}
\item Sa chaîne audio sur SoundCloud : \href{https://soundcloud.com/karlheinz-essl}{https://soundcloud.com/karlheinz-essl}
\end{itemize}

\subsubsection*{Liens pour télécharger les programmes mentionnés}

\begin{itemize}
\item La \emph{Lexikon-Sonate} : \href{http://www.essl.at/works/Lexikon-Sonate.html}{http://www.essl.at/works/Lexikon-Sonate.html}
\item La \emph{RTC-lib} : \href{http://www.essl.at/works/rtc.html}{http://www.essl.at/works/rtc.html}
\item Site officiel de MAX/MSP : \href{https://cycling74.com/products/max/}{https://cycling74.com/products/max/}
\end{itemize}

\subsubsection*{Liens pour écouter quelques pièces mentionnées}

\begin{itemize}
\item Un extrait (enregistré) de la \emph{Lexikon-Sonate} : \\ \href{https://soundcloud.com/karlheinz-essl/lexikon-sonate}{https://soundcloud.com/karlheinz-essl/lexikon-sonate}
\item \emph{Helix 1.0}, quatuor à cordes d'inspiration algorithmique mais \guill{hermétique} et composé sans ordinateur : \\ \href{https://soundcloud.com/karlheinz-essl/helix}{https://soundcloud.com/karlheinz-essl/helix}
\item Performances live réalisées avec le méta-instrument \maze~: \\ \href{https://soundcloud.com/karlheinz-essl/sets/maze}{https://soundcloud.com/karlheinz-essl/sets/maze}
\item Extraits (enregistrés) du programme de type installation \emph{WebernUhrWerk}, conçu en hommage à Anton Webern, générant de la musique d'inspiration sérielle : \\ \href{https://soundcloud.com/karlheinz-essl/webernuhrwerk}{https://soundcloud.com/karlheinz-essl/webernuhrwerk}
\item \emph{Entsagung}, première pièce avec musique électronique \emph{live} créée par Essl grâce à la Station d'informatique musicale : \\ \href{https://soundcloud.com/karlheinz-essl/entsagung}{https://soundcloud.com/karlheinz-essl/entsagung}
\end{itemize}

\subsubsection*{Quelques sites consacrés à la musique algorithmique}

\begin{itemize}
\item \emph{flexatone.org}, site du compositeur contemporain Christopher Ariza : \\ \href{http://www.flexatone.org/}{http://www.flexatone.org/}
\item \emph{musicalgorithms}, site créé à l'initiative du compositeur et chercheur Jonathan Middleton : \\ \href{http://musicalgorithms.ewu.edu/}{http://musicalgorithms.ewu.edu/}
%\item : \\ \href{}{}
%\item : \\ \href{}{}
%\item : \\ \href{}{}
\end{itemize}

\newpage
\section{Liste des œuvres de Karlheinz Essl}

La liste ci-dessous recense chronologiquement toutes les œuvres de Karlheinz Essl (partitions, installations, programmes\dots~mais pas les performances). Les œuvres en \textbf{gras} sont celles qui sont mentionnées dans le document principal. Les œuvres indiquées par un rond ($\bigcirc$) sont les partitions \guill{classique}, celles qui sont indiquées par un carré ($\Box$) utilisent de la musique électronique, et celles indiquées par un triangle ($\rhd$) ont recours à l'informatique (et éventuellement à la musique électronique) pour leur exécution.

\begin{itemize}
\item[$\bigcirc$] \emph{O rosa bella} (1981-1996)
\item[$\bigcirc$] \emph{Conversations} (1983-1984)
\item[$\bigcirc$] \emph{Memento mori} (1984-1985)
\item[$\Box$] \textbf{\emph{Con una certa espressione parlante}} (1985)
\item[$\bigcirc$] \emph{Streichquartett 1985} (1985)
\item[$\bigcirc$] \textbf{\emph{BWV 1007a}} (1986)
\item[$\Box$] \emph{Carambol} (1986)
\item[$\bigcirc$] \textbf{\emph{Helix 1.0}} (1986)
\item[$\bigcirc$] \emph{Helix 2.0} (1986)
\item[$\Box$] \emph{Orgue de Cologne} (1986)
\item[$\Box$] \textbf{\emph{In the Cage}} (1987)
\item[$\bigcirc$] \emph{met him pike trousers} (1987)
\item[$\bigcirc$] \emph{Proportional Circles 2314} (1987)
\item[$\bigcirc$] \emph{Oh tiempo tus piramides} (1988-1989)
\item[$\rhd$] \textbf{\emph{COMPOSE for xLOGO}} (1988-1993)
\item[$\bigcirc$] \emph{\dots~et consumimur igni} (1989-1990)
\item[$\bigcirc$] \emph{Rudiments} (1989-1990)
\item[$\bigcirc$] \emph{Abolition\dots~} (1989)
\item[$\bigcirc$] \emph{Close the Gap!} (1990)
\item[$\Box$] \emph{Zungenreden} (1990)
\item[$\rhd$] \emph{Entsagung} (1991-1993)
\item[$\bigcirc$] \emph{Partikel-Bewegungen} (1991-)
\item[$\bigcirc$] \emph{In Girum. Imus. Nocte} (1991)
\item[$\bigcirc$] \emph{In's Offene!} (1991)
\item[$\bigcirc$] \emph{Space Art Transmission} (1991)
\item[$\rhd$] \emph{Klanglabyrinth} (1992-1995)
\item[$\rhd$] \textbf{\emph{Lexikon-Sonate}} (1992-2007)
\item[$\bigcirc$] \emph{Rapprochement} (1992)
\item[$\rhd$] \emph{Amazing Maze} (1993-2014)
\item[$\rhd$] \textbf{\emph{RTC-lib}} (1993-2015)
\item[$\bigcirc$] \emph{Déviation} (1993)
\item[$\bigcirc$] \emph{Cross the Border} (1994-1995)
\item[$\bigcirc$] \emph{Intervention} (1995)
\item[$\rhd$] \emph{Interferences} (1996-1997)
\item[$\rhd$] \emph{Lexikon-Oracle} (1996-1998)
\item[$\rhd$] \emph{Lexikon-Projekt} (1996-1998)
\item[$\rhd$] \emph{MindShipMind} (1996-1998)
\item[$\bigcirc$] \emph{\dots~wird sichtbar am Horizont} (1996)
\item[$\bigcirc$] \emph{absence} (1996)
\item[$\bigcirc$] \emph{Préambule – Pierrot/Arlequin} (1996)
\item[$\bigcirc$] \emph{à trois – seul} (1997-1988)
\item[$\bigcirc$] \emph{elision} (1997)
\item[$\rhd$] \emph{Lexikon-Sonate online\dots} (1997)
\item[$\rhd$] \textbf{\emph{fLOW}} (1998-2001)
\item[$\rhd$] \emph{Champ d'Action} (1998)
\item[$\rhd$] \emph{Karlheinz Essl's Playing Strategies} (1998)
\item[$\rhd$] \emph{Lexicon-Lecture} (1998)
\item[$\bigcirc$] \emph{mise en scène} (1998)
\item[$\rhd$] \emph{da braccio} (1999-2000)
\item[$\bigcirc$] \emph{four2eight} (1999-2000)
\item[$\Box$] \emph{onwards} (1999-2000)
\item[$\rhd$] \emph{more or less} (1999-2007)
\item[$\rhd$] \textbf{\maze}~(1999-)
\item[$\rhd$] \emph{conVex} (1999)
\item[$\rhd$] \textbf{\emph{Replay PLAYer}} (2000-2001)
\item[$\rhd$] \emph{es wird} (2000)
\item[$\rhd$] \emph{fLOWER} (2000)
\item[$\rhd$] \emph{The Untempered Piano} (2000)
\item[$\rhd$] \emph{BREAKAWAYS} (2001-2002)
\item[$\bigcirc$] \emph{downside up} (2001-2002)
\item[$\rhd$] \emph{ALLgebrah. Eine Kopfweit} (2001)
\item[$\Box$] \emph{GRIDS} (2001)
\item[$\rhd$] \emph{ON FIRE} (2001)
\item[$\rhd$] \emph{Temps fugitive} (2001)
\item[$\bigcirc$] \emph{upward, behind the onstream it mooned} (2001)
\item[$\rhd$] \emph{Sonnez la cloche!} (2002-2003)
\item[$\Box$] \emph{SPACIAL EDITION} (2002-2003)
\item[$\rhd$] \emph{à la recherche de la voix perdue} (2002)
\item[$\rhd$] \emph{DISTANCE OF THE MOON} (2002)
\item[$\bigcirc$] \emph{ex machina} (2002)
\item[$\rhd$] \emph{PENDENTE} (2002)
\item[$\Box$] \emph{Sieben Tore ins Land} (2002)
\item[$\rhd$] \emph{Seelewaschen} (2003-2004)
\item[$\rhd$] \textbf{\emph{Gold.Berg.Werk}} (2003-2016)
\item[$\bigcirc$] \emph{blur} (2003)
\item[$\Box$] \emph{carl mayer, scenar(t)ist – berlin} (2003)
\item[$\rhd$] \emph{Doors / Vrata} (2003)
\item[$\rhd$] \emph{Le mystère d'orgue} (2003)
\item[$\rhd$] \emph{Segreto spaziale} (2003)
\item[$\bigcirc$] \emph{STREAMING} (2003)
\item[$\rhd$] \emph{Father Earth} (2004-2005)
\item[$\Box$] \emph{tides} (2004-2006)
\item[$\rhd$] \emph{action rituelle} (2004)
\item[$\Box$] \emph{blurB} (2004)
\item[$\Box$] \emph{dance:storm} (2004)
\item[$\rhd$] \emph{el-emen'} (2004)
\item[$\bigcirc$] \emph{Faites vos jeux!} (2004)
\item[$\rhd$] \textbf{\emph{FontanaMixer}} (2004)
\item[$\Box$] \emph{Nach viermal geht die Sonne auf} (2004)
\item[$\rhd$] \emph{ULURU} (2004)
\item[$\rhd$] \emph{nature / morte} (2005-2006)
\item[$\Box$] \emph{sculpture} (2005-2006)
\item[$\rhd$] \emph{colorado} (2005-2008)
\item[$\bigcirc$] \textbf{\emph{WebernSpielWerk}} (2005-2015)
\item[$\rhd$] \emph{and thank you I don't think we will meet again} (2005)
\item[$\rhd$] \emph{Balkonszene} (2005)
\item[$\Box$] \emph{Kalimba} (2005)
\item[$\rhd$] \emph{Rouge de Rouge} (2005)
\item[$\rhd$] \emph{Through the Curly Rain-Taps} (2005)
\item[$\rhd$] \textbf{\emph{WebernUhrWerk}} (2005)
\item[$\bigcirc$] \emph{missing electronic part of Spiegel III} (2006-2007)
\item[$\rhd$] \emph{Von Hirschen und Röhren} (2006-2007)
\item[$\bigcirc$] \emph{7x7} (2006-2009)
\item[$\rhd$] \emph{(7x7)\^{}7} (2006)
\item[$\rhd$] \emph{AIR BORNE} (2006)
\item[$\rhd$] \emph{Berliner Luft} (2006)
\item[$\rhd$] \emph{Deconstructing Mozart} (2006)
\item[$\Box$] \emph{Drowned} (2006)
\item[$\Box$] \emph{inside/out} (2006)
\item[$\rhd$] \emph{Leimung} (2006)
\item[$\rhd$] \emph{Oracle Night} (2006)
\item[$\rhd$] \emph{Panta Rhei} (2006)
\item[$\rhd$] \emph{blurred} (2007-2011)
\item[$\bigcirc$] \emph{Cinq} (2007)
\item[$\rhd$] \emph{FRÄULEIN ATLANTIS} (2007)
\item[$\rhd$] \emph{Stimmen.Hören} (2007)
\item[$\rhd$] \emph{WebernWeben} (2007)
\item[$\rhd$] \emph{BOCHUMSTEP} (2008-2009)
\item[$\rhd$] \emph{non Sequitur} (2008-2009)
\item[$\bigcirc$] \emph{While my guitars gently whip} (2008-2009)
\item[$\rhd$] \textbf{\emph{Sequitur}} (2008-2010)
\item[$\bigcirc$] \emph{Listen Thing} (2008)
\item[$\rhd$] \emph{KlangDerWisch} (2009-2015)
\item[$\bigcirc$] \emph{Chemi(s)e} (2009)
\item[$\rhd$] \emph{Demo Crazy} (2009)
\item[$\bigcirc$] \emph{Detune} (2009)
\item[$\rhd$] \emph{Pandora's Secret} (2009)
\item[$\bigcirc$] \textbf{\emph{Take the C Train}} (2009)
\item[$\rhd$] \emph{LABoratorio} (2010-2011)
\item[$\Box$] \emph{Sterbebett mit Pappendeckeln} (2010-2011)
\item[$\bigcirc$] \emph{Hypostasis} (2010)
\item[$\rhd$] \emph{Proper Ties} (2010)
\item[$\rhd$] \emph{whatever shall be} (2010)
\item[$\rhd$] \textbf{\emph{juncTions}} (2011-2012)
\item[$\rhd$] \emph{Familiar Strangers} (2011)
\item[$\bigcirc$] \emph{Miles to go} (2012)
\item[$\rhd$] \emph{Si!} (2012)
\item[$\rhd$] \emph{Suspended Suspense} (2012)
\item[$\bigcirc$] \emph{under wood} (2012)
\item[$\Box$] \emph{not yet over} (2013)
\item[$\rhd$] \emph{Pachinko} (2013)
\item[$\rhd$] \emph{Parsifal-Kristall} (2013)
\item[$\bigcirc$] \textbf{\emph{STERN}} (2013)
\item[$\rhd$] \emph{Tritan's Lament} (2013)
\item[$\rhd$] \emph{WalkürenWalk} (2013)
\item[$\rhd$] \emph{Omnia in omnibus} (2014-2015)
\item[$\Box$] \emph{Herbeck extended} (2014)
\item[$\rhd$] \emph{Herbecks Verspechen} (2014)
\item[$\rhd$] \emph{Natura naturans} (2014)
\item[$\bigcirc$] \emph{RESONAVIT} (2014)
\item[$\Box$] \emph{Über Österreich – Juwele des Landes} (2014)
\item[$\bigcirc$] \emph{VIRIBVS VNITIS} (2014)
\item[$\rhd$] \emph{Autumn's leaving} (2015)
\item[$\rhd$] \emph{GO MATA GO} (2015)
\item[$\Box$] \emph{imagination} (2015)
\item[$\bigcirc$] \emph{Mozart-Lamento} (2015)
\item[$\rhd$] \emph{exit*glue} (2016)
\item[$\bigcirc$] \emph{river\_run} (2016)
\end{itemize}

\newpage
\section{Chronologie}

Cette section présente une chronologie succincte du parcours d'Essl.

\begin{tabular}{rl}
\textbf{1960} & Naissance \\
\textbf{Années 1960} & Apprentissage du piano \\
\textbf{Années 1970} & Apprentissage de la contrebasse et de la guitare électrique, goût pour le \\ & rock avant-gardiste \\
\textbf{1975} & Découverte Stockhausen à travers le groupe de rock \emph{CAN} \\
\textbf{1981-1987} & Études à l'Académie de musique et des arts du spectacle de Vienne \\
\textbf{1983} & Première composition : \emph{Conservations} \\
\textbf{1985} & Mise au point de l'idée de \emph{Strukturgeneratoren} \\
\textbf{1985} & Acquisition d'un premier ordinateur : l'Atari ST \\
\textbf{1986} & Premières compositions d'inspiration algorithmique (\emph{BWV 1007a}, \emph{Helix 1.0}) \\
\textbf{1985} & Rencontre de Gottfried Michael Koenig à l'Institut de Sonologie \\
\textbf{1988} & Rencontre de John Cage \\
\textbf{1988-1993} & Élaboration de la bibliothèque \emph{COMPOSE} dans le langage xLOGO \\
\textbf{1990-1994} & Participation aux \emph{Darmstädter Ferienkurse für Neue Musik} \\
\textbf{1991} & Soutenance d'une thèse de musicologie, \emph{Das Synthese-Denken bei} \\ & \emph{Anton Webern} \\
\textbf{1992} & Stage d'informatique musicale à l'IRCAM \\
\textbf{1992} & Rédaction d'un essai sur le constructivisme radical et ses implications en \\ & musique, \emph{Kompositorische Konsequenzen des radikalen Konstruktivismus} \\
\textbf{1992-2007} & Composition/programmation de la \emph{Lexikon-Sonate} \\
\textbf{1993-} & Élaboration de la bibliothèque \emph{RTC-lib} dans le langage MAX/MSP \\
\textbf{1995-} & Enseignement de la composition, à Linz jusqu'en 2006, puis à Vienne \\
\textbf{1997} & Présentation avec concert-portrait au festival de Salzbourg \\
\textbf{1997} & Acquisition d'un Apple PowerBook G3, d'une puissance comparable à la \\ & Station d'informatique musicale de l'IRCAM \\
\textbf{1998-2001} & Projet \emph{fLOW}, mettant l'accent sur l'improvisation et la diversité des collabo- \\ & rations artistiques \\
\textbf{1999-} & Développement de l'instrument virtuel \maze \\
\textbf{1999-2016} & Poste de conservateur au Essl Museum
\end{tabular}

\newpage
\section{Entretien avec Karlheinz Essl}

Cet entretien a été réalisé au studio du compositeur, dans le Essl Museum, en deux fois, les 27 et 28 avril 2016. La transcription et la traduction depuis l'allemand ont été réalisées grâce à l'aide de Jana Gulyas et Ferdinand Schlie.

\subsection{Divers}

\textbf{RV ---} Quel genre de musique aimez-vous écouter ?

\textbf{KHZ ---} J'écoute beaucoup de musique, avant tout pour des raisons professionnelles. Comme j'ai souvent affaire à des étudiants, je m'informe beaucoup sur ce qui est composé et produit actuellement. C'est de l'écoute pour des raisons professionnelles. J'en fais en permanence, et j'écoute évidemment ainsi des choses très variées. En premier lieu de la Nouvelle Musique bien sûr, de la musique électronique, de la musique expérimentale\dots~Mais j'écoute aussi de la musique pour mon loisir ! Et en fait ce sont des choses très lointaines de celles que je fais moi-même. C'est surtout la musique que j'écoutais plus jeune, adolescent ou jeune adulte. Pendant un moment je me suis intéressé de manière très intense avec certaines formes de rock avant-gardiste, et c'est ce que j'aime toujours écouter. Par exemple\dots~\emph{Gentle Giant}, ou \emph{Can}, qui est l'un de mes groupes préférés, c'étaient des représentants du \emph{Krautrock}, qui ont collaboré avec Stockhausen. Plus précisément, certains d'entre eux étaient élèves de Stockhausen, et c'est à partir de ce groupe que j'ai découvert Stockhausen, ce qui a été le pont qui m'a lancé sur les rails de la musique électronique.

\textbf{RV ---} Pourriez-vous me raconter votre découverte de l'œuvre de Georges Perec ?

\textbf{KHZ ---} Je me souviens que c'était encore pendant mes études, dans les années 80. Je m'étais intéressé à toutes les formes possibles d'algorithmes, et à la génération de structures musicales, et aussi à la littérature expérimentale, qui travaille avec des méthodes semblables. Il y a en Autriche une formation qui s'est réunie à Vienne après la Seconde Guerre Mondiale, dans les années 50, qui s'appelait le \emph{Wiener Gruppe} [Groupe de Vienne]. Des auteurs autrichiens très connus y prenaient part, comme H.C. Artmann, Konrad Bayer, Gerhard Rühm\dots~Ils avaient une approche très structurelle de la littérature, avec des méthodes modernes pour déconstruire ou construire des textes. Et dans ce contexte, quelqu'un m'a informé de l'existence en France, à Paris, d'un groupe similaire, l'OuLiPo, qui travaillait avec le même genre d'idées. C'était un peu plus tard mais ils sont sûrement influencé Artmann d'une certaine manière. Et c'est à partir de cette information que j'en suis venu à George Perec. J'ai lu beaucoup de ses ouvrages, et me suis beaucoup intéressé pour ses travaux pendant un moment. Bien que je ne parle pas français, j'ai lu \emph{La disparition} en édition bilingue, pas en entier, mais toujours avec la traduction allemande et l'original français en parallèle. Et je me suis rendu compte que c'était incroyablement bien traduit !

\textbf{RV ---} Vous devez donc parler un peu français !

\textbf{KHZ ---} Oui, un peu, j'ai aussi vécu en France pendant un moment, lorsque j'ai travaillé à Paris à l'IRCAM. Je sais dire des choses élémentaires, mais pas vraiment tenir une conversation. Et puis je peux lire un peu. Et donc il y avait cette traduction géniale d'Eugen Helmlé, qui est bilingue, franco-allemand, et j'étais très impressionné qu'il soit possible de traduire un libre aussi compliqué, un livre sans la lettre E, qui est la lettre la plus courante des langues française et allemand, et de traduire aussi sans la lettre E. C'est ce genre de choses qui m'ont fasciné chez Perec.

\textbf{RV ---} Quand avez-vous utilisé un ordinateur pour la première fois, et quand avez-vous appris à programmer ?

\textbf{KHZ ---} C'était pendant mes études, dans les années 80. J'ai commencé à étudier la musicologie en 1979, et j'ai passé le concours d'entrée de l'Académie de Composition en 1981. Et pendant mes études de musicologie j'avais un ami, que j'ai toujours, Gerhard Eckel, qui depuis est devenu professeur d'informatique musicale à l'université de Graz. Et Gerhard a commencé à travailler avec l'ordinateur très tôt. Nous avions dans notre institut de musicologie une section dédiée aux recherches sur le son [\emph{Klangforschung}, en anglais \emph{Sound Studies}]. Ils avaient un labo dans l'Académie, où ils travaillaient déjà à l'époque avec un \emph{mainframe computer}, c'est-à-dire d'énormes calculateurs, et Gerhard s'est impliqué là-dedans dès le départ. Il a aidé et travaillé avec les chercheurs, puis il a commencé à programmer, à appris le C, très tôt. Je me souviens que pour ma part j'ai acheté mon premier ordinateur en 1985. Auparavant je rejetais complètement l'ordinateur. Pendant mes premières années d'études, de 1981 à 1983, j'ai eu un professeur de théorie de la musique très traditionnel, et je ne m'intéressais qu'à la vieille musique. Le contrepoint, les motets, la musique médiévale\dots~J'ai beaucoup étudié cela pendant mes études de musicologie, ça me fascinait, en particulier la musique du Moyen-Âge central : Perotin, l'école française de Notre-Dame du \XIIIe~siècle\dots~Et puis Gerhard m'a dit qu'avec l'ordinateur on pouvait faire plein de choses intéressantes, ce à quoi je lui ai répondu qu'il n'en était pas question, sur un ordinateur tout est réglé d'avance, ce n'est pas intéressant\dots~Et nous avons fini par décider de créer ensemble un groupe de compositeurs, d'organiser des concerts communs. Je m'étais déjà intéressé pour la musique électronique à cette époque, la connexion s'était faite par Stockhausen. Alors Gerhard et moi avons suivi une formation sur la musique électronique --- à l'époque tout était encore analogique, sans ordinateur, avec des bandes magnétiques. Et nous avons donc fondé notre groupe à cette occasion, avec le super nom de \guill{\emph{Kybernikos}}. L'un d'entre nous était testeur pour un magazine informatique, et avait toujours chez lui les ordinateurs dernier cri. Il faut imaginer que c'était en 1984-1985, les PC venaient de sortir et n'étaient pas encore très répandus. Un été cet ami m'a demandé de garder sa maison pendant un mois, d'arroser les plantes, et il a ajouté que je pouvais travailler avec son ordinateur. C'était un ordinateur de bureau Siemens. J'ai commencé à écrire ma thèse dessus, j'ai découvert à quel point c'était pratique d'utiliser l'ordinateur comme une machine ) écrire, à quel point il est plus facile de rédiger et de modifier des textes. Quand mon ami est rentré chez lui, je lui ai dit que j'étais devenu complètement dépendant de l'ordinateur, que j'en voulais un moi aussi, et je lui ai demandé ce que je devais acheter. Il m'a conseillé d'attendre encore deux mois, et la sortie d'un nouvel appareil qui arrivait sur le marché, Atari. Un modèle qui, au fond, dérivait de l'Apple Macintosh. %!!!Apple gab es schon früher, aber Apple Macintosh!!!
L'Apple était à l'époque beaucoup trop cher pour un étudiant. L'Atari coûtait le tiers --- ce qui était déjà beaucoup d'argent ---, il avait une interface graphique, et je l'ai acheté. Le problème, c'est qu'il n'y avait pas de logiciels. Tout ce qu'il comportait, c'étaient un jeu d'aventure (en texte, une fiction interactive) et le BASIC. Mon ami Gerhard a alors dit que, maintenant que j'avais un ordinateur, nous pouvions commencer à faire des expériences. Comme jm'étais déjà intéressé aux algorithmes et à la musique sérielle, nous avons commencé à générer des structures particulières en BASIC, à écrire de petits programmes, et je me suis rendu compte que c'était passionnant. Peu après j'ai découvert le LOGO, un langage de programmation. C'est une invention de Marvin Minsky, un chercheur en informatique très célèbre du MIT, et du pédagogue et psychologue suisse Jean Piaget. C'était une collaboration géniale. Piaget, qui était donc un pédagogue, s'est demandé comment on pouvait mettre des outils dans les mains des enfants avec lesquels ils puissent appréhender ou construire leur environnement, ce qui a mené à l'idée des micromondes. Le logiciel --- Piaget a fini par en arriver au logiciel --- comme moyen d'accès à la connaissance, ou pour analyser des problèmes. La compréhension ou l'analyse d'un problème permet d'arriver à une solution, et on peut programmer une telle solution, ce qui revient à dire construire l'outil pour résoudre le problème. Et l'idée de Marvin Minsky et de Piaget était d'inventer un langage de programmation simple et intuitif, qui permette aux enfants (ou aux adultes) qui l'utilisent d'écrire leurs propres programmes et de pouvoir réutiliser ces codes et ces programmes. C'était un langage interprété, bien sûr. Et l'un des instruments de sortie était cette \guill{\emph{turtle}}, une sorte de pointeur de souris, ou plutôt l'opposé de la souris que l'on déplace à la main puisque cette tortue était contrôlée par l'ordinateur. Elle a un crayon dans la gueule et se déplace sur une feuille de papier ; et on peut programmer son mouvement pour qu'elle dessine ainsi des figures géométriques (ou non-géométriques). C'était fondamentalement un outil graphique, mais très intuitif, haptique, tactile, que l'on pouvait manipuler, regarder, et tout cela a rendu le LOGO célèbre. J'avais un interpréteur LOGO sur mon Atari, qui était mauvais, mais peu après deux mathématiciens allemands ont mis sur le marché une implémentation qui s'appelait xLOGO, \emph{experimental LOGO}. Ce langage m'a beaucoup impressionné, parce que j'ai immédiatement eu le sentiment de pouvoir construire mes propres micromondes, en rapport avec la composition musicale. J'ai commencé à coder divers formalismes et algorithmes en LOGO, et pendant au moins dix ans j'ai composé mes morceaux avec. Il faut noter qu'en 1985 j'étais en contact avec Gottfried Michael Koenig, un pionnier de l'informatique musicale. Je l'ai parce que mon ami Gerhard Eckel avait fait un semestre d'échange à Utrecht, à l'Institut de Sonologie. Koenig y était professeur. Quand j'ai rendu visite à Gerhard, je suis allé faire un tour à la bibliothèque, j'ai cherché des partitions, et j'ai trouvé un quatuor de Gottfried Michael Koenig ; je l'ai recopié et suis rentré chez moi avec la copie pour l'analyser. J'aimais beaucoup l'analyse musicale, j'ai étudié un peu de tout, beaucoup de pièces sérielles, l'école de Vienne, et là j'ai essayé d'analyser cette pièce de Koenig, et je n'arrivais à rien. Je lui ai écrit une lettre pour lui demander s'il pouvait me fournir quelques pistes. Une semaine plus tard, il m'a retourné une lettre très épaisse dans laquelle il m'expliquait des choses fondamentales, en particulier : \guill{cela ne m'étonne pas que vous n'arriviez pas à analyser le morceau, parce que je l'ai composé avec des opérations aléatoires}. C'était ma première rencontre avec le hasard en musique. Le plus fort, c'est que, quand on écoute la pièce, elle n'a pas du tout l'air d'être faite au hasard. Ça m'a beaucoup intéressé, ça me travaillait en permanence, et j'ai repris et implémenté dans mon système LOGO beaucoup des concepts, des idées, des fonctions que Koenig a développés, notamment dans ses premiers programmes comme \emph{Projekt 1}, qui est pour ainsi dire un classique de la composition algorithmique. Ainsi j'ai conçu une boîte à outils avec laquelle j'ai écrit un grand nombre de pièces, des pièces instrumentales et non électroniques. Et puis, entre 1991 et 1993, j'ai été à l'IRCAM à Paris, où ils mettaient alors au point la Station d'informatique musicale. C'était un ordinateur NeXT avec une carte son spéciale et un super processeur, qui pouvait faire du \emph{Soundprocessing}, synthèse et manipulation de sons, en temps réel. C'était l'œuvre d'un informaticien italien du nom de Giuseppe di Giugno, qui à l'origine a développé la machine 4X, au départ un PDP-11 avec un processeur dédié au son. Et quand la machine NeXT est sortie, j'ai dit : ça, c'est l'avenir, c'est portable, le processeur est petit, on peut enregistrer des concerts avec. Ils ont même développé avec elle l'électronique pour la pièce \emph{Répons} de Pierre Boulez. Comme cette station d'informatique musicale était en quelque sorte à un stade bêta, l'IRCAM a invité plusieurs compositeurs, de jeunes compositeurs du monde entier, à écrire une pièce avec elle. Ils voulaient la tester. J'ai eu la chance d'être au bon endroit au bon moment. J'ai suivi un cours d'été à l'IRCAM où je me suis familiarisé avec beaucoup de choses, les logiciels, les studios, ainsi que le langage de programmation MAX, qui tournait alors exclusivement sur la Station de l'IRCAM, dans une version très simple qui ressemblait plus à PD qu'à l'actuelle MAX. Je l'ai utilisé pour ma pièce, et j'étais accroché à ce MAX, parce que j'ai immédiatement vu la possibilité qu'il ouvrait, celle d'utiliser en temps réel les générateurs \guill{hors du temps} que j'avais développé pendant des années sur mon Atari. Je pouvais entendre les résultats sur le champ, et pendant le temps même de l'exécution, intervenir, modifier des paramètres, et le son changeait en conséquence. Je me suis fait la remarque que c'était comme si l'ordinateur était soudain devenu un instrument. Quand je suis rentré chez moi en 1993, je me suis fait la réflexion que malgré cette situation extraordinaire que j'avais eue à l'IRCAM, je ne pouvais rien faire de tel chez moi. Je n'avais pas les moyens de m'acheter une Station d'informatique musicale ! Jean-Baptiste Barrière m'a alors dit d'attendre cinq ans, que cinq ans plus tard j'aurais l'équivalent sur mon ordinateur personnel. Pour moi, c'était impossible, inimaginable. Mais effectivement, en 1998, un Apple avec un processeur G3, un processeur Motorola, très rapide par rapport aux habitudes d'alors, est sorti. On pouvait sans matériel spécifique au traitement du son, directement avec le CPU, calculer le son en temps réel. MAX a immédiatement connu une extension, MAX/MSP. Il y avait tout ce que j'avais connu sur la Station d'informatique musicale. Je pouvais utiliser mon Powerbook Pro sans matériel supplémentaire. Je l'ai encore, un vieil engin noir. Il y a peu je l'ai ressorti et rallumé parce que je cherchais un fichier précis, et il fonctionnait toujours. Il a presque dix-huit ans, et il fonctionne encore ! En résumé, ça s'est passé ainsi : je suis allé du rock à la musique sérielle, de la musique sérielle à la musique électronique, de la musique électronique à l'informatique musicale, et dans celle-ci, de la génération de partition à la musique électronique \emph{live}. On peut voir cela comme tout un processus, de trente ans.

\textbf{RV ---} Est-ce qu'au cours de ce processus le temps réel a été un rêve avant d'être effectivement possible ?

\textbf{KHZ ---} Non, pas du tout ! Je me suis toujours demandé pourquoi on parlait autant du temps réel. Peu importe, me disais-je, si je dois attendre une nuit pour que le calcul termine et m'imprime une page, une partition, un tableau que je dois ensuite transformer en notation, mais\dots~Le premier programme que j'ai écrit avec MAX est la \emph{Lexikson-Sonate}. C'était en réalité un test, je voulais tester MAX. Que peut-on composer avec ce langage ? J'ai alors essayé de mettre au jour des structures musicales avec MAX (à l'époque c'était en MIDI, avec un \emph{sampler} de piano connecté), que l'on peut jouer au piano. Et j'ai vu que cela fonctionnait en temps réel, que je pouvais entendre jouer ce que je codais. Et je pouvais faire des modifications, et immédiatement l'harmonie ou la vitesse changeaient. C'est alors que j'ai compris ce que cela signifiait de faire de la musique en temps réel.

\textbf{RV ---} Vous avez donc codé en xLOGO puis en MAX ; j'ai aussi lu quelque part que vous aviez aussi un peu travaillé avec Perl et Javascript. Est-ce que vous avez essayé encore d'autres langages de programmation ?

\textbf{KHZ ---} Oui. J'ai un peu essayé SuperCollider. Je trouve que c'est un super langage, mais je n'ai pas vraiment eu le temps de le pratiquer, cela m'intéresserait beaucoup. Et ce que je trouve très enthousiasmant aujourd'hui est Opusmodus. C'est une nouvelle génération d'environnements de composition algorithmique, qui a été développée par un collectif de musiciens français et polonais. C'est sur le marché depuis un an. Ça s'appuie sur LISP, et possède une implémentation Music XML extrêmement puissante. Avec ce système, on peut créer de vraiment très belles partitions. C'est un langage très mûrement réfléchi, avec lequel on peut directement créer des partitions grâce à des algorithmes (ou sans). Il y a toutes sortes de sorties : MIDI, son, XML, partition, n'importe quel format graphique\dots~C'est très ouvert et facile à configurer, et il y a un une immense bibliothèque de fonctions que l'on peut réutiliser ou développer. J'aimerais m'intéresser de près à ce langage cet été. Pendant les vacances j'ai déjà pris quelques jours pour l'essayer, j'ai pris contact avec le développeur, je l'ai invité à Vienne et il est venu à l'université. J'aimerais bien utiliser Opusmodus prochainement, autrement dit revenir quasiment à la création de partition, ce que je n'ai pas fait depuis longtemps, parce que je fais presque tout en temps réel, avec de la musique électronique \emph{live}, mais le reste m'intéresse à nouveau.

\textbf{RV ---} Encore une question sur vos études : vous avez étudié la chimie\dots

\textbf{KHZ ---} C'était à l'\emph{Oberstufe Gymnasium}, avec spécialité chimie. C'est un an plus long que le lycée et le diplôme correspond en quelque sorte à ingénieur en chimie, mais pas au niveau universitaire. Au moins, c'était un moment génial, parce que j'ai accumulé beaucoup de bagage en sciences de la nature.

\textbf{RV ---} Et vous avez décidé de faire de la musicologie juste après ce lycée particulier ?

\textbf{KHZ ---} Exactement. J'ai décidé d'étudier la musicologie, contre la volonté de mes parents, qui évidemment voulaient que j'apprenne quelque chose de plus \guill{convenable}. Je me suis toujours intéressé à la musique, pendant ma jeunesse j'en faisais tout le temps, j'ai joué dans des groupes, fait des concerts en famille, etc. Je n'ai jamais su exactement quoi faire, mais la musicologie était ce qu'il y avait de plus proche, alors j'ai repris des cours de piano auprès de mon ancienne professeur, qui m'a dit que je devais étudier la composition, la musicologie, que c'était très scientifique. Je ne savais pas si j'y arriverais, mais elle m'a dit d'essayer, m'a recommandé de me présenter de sa part au professeur Uhl. C'est ce que j'ai fait, il m'a accueilli, il était déjà très vieux, et très amical, il a regardé des morceaux que j'avais composés et a dit que je pouvais passer le concours d'entrée. J'ai été pris. Ça s'est passé facilement, en fait !

\textbf{RV ---} Et vos parents n'étaient pas d'accord ? J'ai pourtant l'impression qu'ils s'intéressent à l'art !

\textbf{KHZ ---} Oui, beaucoup ! Mais à l'époque ce n'était pas encore comme aujourd'hui. À ce moment-là ils ont commencé leur collection, mais dans des domaines très réduits, sans musée, très modestement.

\textbf{RV ---} Comment décririez-vous votre style musical ?

\textbf{KHZ ---} Il a beaucoup évolué. Dans les années 90, après mes études, j'étais très influencé par le sérialisme, mais plutôt dans le sens de Boulez, le prolongement du sérialisme. J'écrivais beaucoup de musique de chambre et de musique d'ensemble, complexe, inspirée par la \emph{New Complexity}. Ça me fascinait, ce type d'esthétique. Et puis en 1997 j'ai été en quelque sorte présenté en tant que compositeur au festival de Salzbourg, en tant que \guill{\emph{next generation}}. J'ai eu droit à un concert-portrait avec de super ensembles, de super chefs\dots~J'avais 37 ans et je me suis dit : c'est une apogée ! Tout s'est extrêmement bien passé, mais quand tout fut fini, je suis d'une certaine manière tombé au fond d'un trou profond. Je me suis demandé : \guill{Alors voilà, c'est tout ? Est-ce que c'est une vie ?. S'asseoir devant son bureau, écrire pendant des semaines des morceaux qui seront peut-être jouées un jour, une ou deux fois, puis à nouveau, à la maison, écrire des morceaux, des partitions. Plus jeune, quand je jouais dans des groupes, je donnais au moins un concert par mois. Et nous faisions beaucoup de répétitions, d'improvisation, nous avons vécu des choses très intéressantes, et aujourd'hui je suis assis seul dans une tour d'ivoire et j'écris mes partitions\dots}. Je me suis dit qu'il fallait que je me sorte de là. Je voulais devenir à nouveau performeur, retourner sur scène, jouer, jouer en \emph{live}. Et puis j'ai réalisé que je ne savais plus vraiment jouer d'un instrument. J'avais joué du piano mais avais abandonné, j'avais étudié la contrebasse, que j'avais abandonnée pour avoir du temps pour composer, avant j'avais joué de la guitare électrique, que j'avais aussi abandonnée depuis longtemps\dots~En résumé, je pouvais jouer de quelques instruments mais n'avais plus de pratique. Et il n'était pas envisageable, moi qui avais atteint un certain niveau en tant que compositeur, de recommencer comme instrumentiste, comme un enfant, faire mes gammes. La solution était de trouver mon propre instrument. Et comme MAX/MSP était sorti, j'ai commencé à \guill{construire} un instrument avec, que j'ai appelé le \maze. J'en joue encore aujourd'hui, ou plutôt son descendant, le programme a évolué sur de nombreuses générations. Et puis à ce moment-là, en 1998, après Salzbourg, j'ai décidé de monter un grand projet, \emph{fLOW}. Ce projet reposait sur un générateur de paysages sonores que j'ai écrit en MAX, sur lequel différents musiciens improvisent. J'ai décliné cela en de nombreuses versions, en invitant de plus en plus de musiciens, avec des origines diverses : le jazz, la musique expérimentale, la Nouvelle Musique, le classique\dots, à faire une performance de \emph{fLOW} avec moi. J'ai ainsi appris ce que c'est que de travailler et d'improviser avec des musiciens de spécialités et de langues et d'idiomes différents. Mon style de compositeur s'est transformé à ce prisme. C'est une musique qui part davantage de l'écoute, qui convient certes toujours des idées constructives et des algorithmes, mais plus de façon hermétique, subtilement pesée à ma table de travail, mais qui provient toujours d'une manière ou d'une autre de l'expérience. Mon style a beaucoup changé avec la fréquentation de la musique électronique en \emph{live} et des algorithmes en temps réel.

\textbf{RV ---} Pour quoi composez-vous ? Quel auditeur imaginez-vous, lorsque vous composez ou improvisez ?

\textbf{KHZ ---} Je n'imagine pas un auditeur précis --- mais je prends toujours l'auditeur en compte. C'est-à-dire que je ne compose pas de manière abstraite. Peut-être qu'avant il en allait un peu autrement, mais aujourd'hui j'ai toujours à l'esprit la façon dont la musique se déploie dans la pièce, quel effet cela produit. Je pense toujours à l'auditeur, y compris pendant les phases d'écriture, comment le temps s'organise, ce qui peut être enregistré ou non, ce que je peux retravailler, ce qui peut me surprendre ou m'ennuyer. Je pense toujours de manière très empirique.

\textbf{RV ---} J'ai aussi lu que vous aimiez beaucoup improviser mais que vous vous interdisez de le faire sans public.

\textbf{KHZ ---} Exactement. C'est un bon point ! J'ai des projets d'improvisation où, comme aujourd'hui avec Agnes [Heginger], nous n'avons jamais répété. L'idée est d'arriver devant le public et de jouer \emph{out of the blue}. Il y a toujours un concept, un cadre, aujourd'hui par exemple c'est Friedrich Nietzsche, il y a aussi eu \guill{poèmes d'amour}\dots~Agnes lit beaucoup, a beaucoup de connaissances littéraires. Elle apporte tout ce qu'elle peut, des livres, regarde toutes sortes de choses, prend des notes. Elle arrive ainsi sur scène avec beaucoup de matériau, mais elle ne sait pas ce qu'elle va utiliser, et je ne le sais pas non plus. Alors commence une sorte de jeu, un jeu devant le public, au cours duquel nous nous renvoyons mutuellement la balle, en nous créons une pièce en temps réel avec ce matériau. Souvent, quand je réécoute les performances (je fais toujours des enregistrements et les écoute à chaque fois), cela sonne comme des compositions qui auraient été répétées. Avec elle cela fonctionne toujours extrêmement bien, parce qu'elle dispose d'un incroyable éventail technique et sensible, elle est très ouverte, elle peut changer les choses très rapidement, par exemple elle entend des sons que je produis et peut immédiatement y réagir. Et réciproquement je connais bien ce qu'elle fait, et je peux immédiatement essayer de trouver une réponse.

\textbf{RV ---} Diriez-vous que votre musique est une musique autrichienne ?

\textbf{KHZ ---} C'est difficile à dire !\dots~Les premiers morceaux que j'ai écrit (enfin, pas mes écrits d'enfance, mais en tant que jeune compositeur) étaient à la recherche d'un prolongement de l'école de Vienne, Schönber, Webern, Berg. Mais je n'ai pas tarder à rejeter cette voie. Ensuite vint le sérialisme et ses concepts très abstraits, qui n'avaient plus grand-chose à voir. Et puis je me suis intéressé à des concepts qui viennent d'autres régions du monde, ceci par exemple [il montre un instrument], c'est une cithare, un \emph{guzheng} chinois, aussi appelé \emph{koto} au Japon. Il y a 21 cordes. Je l'ai accordée traditionnellement, en gamme pentatonique, qui se répète à l'octave toutes les cinq notes. Mais ensuite j'ai changé l'accordage pour qu'il n'y ait pas de répétition à l'octave : on a d'abord ré, fa, sol, la, do, mais la note suivante n'est pas un ré, mais un mi, et ensuite sol, etc. À la fin on obtient le total chromatique, mais à petite échelle on a toujours cette échelle pentatonique, mais déplacée chromatiquement et harmoniquement. Et je viens de publier une pièce avec cet instrument réglé ainsi, et avec de la musique électronique \emph{live}, et le résultat final a peu de rapport avec l'instrument original. Le point de départ fait toujours référence à l'instrument, mais il y a un processus de transformation, d'orchestration considérable.

\textbf{RV ---} Est-ce que vous programmez pour votre vie privée, pour autre chose que vos compositions ?

\textbf{KHZ ---} J'ai été \emph{webmaster} du musée. Et puis j'ai passé la main. D'abord ce sont des étudiants à moi qui s'en sont occupés, maintenant ce sont mes enfants. Mais ce n'est pas vraiment de la programmation\dots~Sinon, pas vraiment. Comme vous l'avez dit, j'avais un peu commencé à étudier le Perl. J'ai un ami en Allemagne, un chercheur en littérature, qui a un projet Internet très intéressant, avec de la littérature générative. Il a aussi implémenté en Perl des travaux de Raymond Queneau comme \emph{Cent mille milliards de poèmes}, il a une super page web qui existe depuis vingt ans, où il rassemble toutes sortes de pièces, de textes, d'algorithmes. Et finalement c'est lui qui m'a aidé à mettre en ligne quelques uns de mes travaux.

\textbf{RV ---} Avez-vous souvent utilisé la littérature pour vos œuvre ? Hormis Borges et Joyce ?

\textbf{KHZ ---} Andreas Okopenko, bien sûr. Son \emph{Lexikon-Roman}, ce roman hypertexte, a été très important pour moi. Il a inspiré ma \emph{Lexikon-Sonate}, que j'ai réalisée quand j'étais à Paris, d'abord seulement comme une étude pour apprendre à maîtriser MAX. Quand je suis rentré à Vienne, un groupe d'artiste et d'informaticiens qui voulait faire une version électronique de ce livre m'a contacté. C'était avant le \emph{World Wide Web}, le lien hypertexte n'existait pas encore ! Je me suis rendu compte que ce que j'avais fait à Paris allait tout-à-fait dans la même direction que ce qu'ils faisaient, et voilà, j'avais un titre pour ma pièce : il y avait eu le \emph{Lexikon-Roman}, et maintenant il y avait la \emph{Lexikon-Sonate}. Et j'ai aussi travaillé avec le \emph{Wiener Gruppe}. À cette époque j'étais en contact avec deux auteurs de ma génération, Ferdinand Schmatz et Franz Josef Czernin, qui ont eux aussi écrit leur propre programme informatique, POE (\emph{Poetic Oriented Evaluations}). Ils ont eux aussi utilisé des \emph{Strukturgeneratoren}, au fond, et inventer des algorithmes, de filtre, de synthèse, avec lesquels on peut produire des textes. Je les ai contactés, nous nous sommes rencontrés plusieurs fois, et depuis nous sommes toujours en relation. Et puis il y a eu celui qui a écrit les scripts Perl pour moi, Florian Cramer, qui a maintenant une chaire de littérature expérimentalle à Rotterdam. Lui aussi est un pionnier de la littérature en réseau\dots~Avant tout il est chercheur en littérature, mais a aussi produit beaucoup de textes avec ces algorithmes. J'ai beaucoup appris de lui. Voilà mes références littéraires. Enfin, j'aime aussi lire des livres normaux !

\subsection{La \emph{Lexikon-Sonate}}

\textbf{RV ---} Pourquoi \guill{\emph{Sonate}} ? \guill{\emph{Lexikon}} fait référence au \emph{Lexikon-Roman} est reflète aussi la structure du morceau, mais pourquoi \guill{\emph{Sonate}} ?

\textbf{KHZ ---} Le titre est d'une certaine manière un jeu de mot, une sorte de calembour. La pièce fait effectivement référence au \emph{Lexikon-Roman} d'Andreas Okopenko, qui a la structure d'un dictionnaire. Ce n'est à l'évidence pas un roman, mais un dictionnaire. Il se produit chez le lecteur une restructuration de l'œuvre au fur et à mesure de la lecture. En ce sens, on peut dire que le genre du roman, qui est conçu comme un genre linéaire montrant la plupart du temps une évolution bien précise, y est revisité de manière ironique. Et donc j'ai choisi pour ma pièce une référence à la sonate, qui est le pendant du roman. Ce n'est pas du tout une sonate, et pourtant elle contient tout ce qu'une sonate peut contenir.

\textbf{RV ---} La forme sonate serait aussi importante que la linéarité pour le roman, et ici tous deux sont absents, c'est bien cela ?

\textbf{KHZ ---} D'une certaine manière --- mais tout y est. Dans le \emph{Lexikon-Roman} il y a des éléments d'un voyage, parce que le roman est aussi par essence une sorte de voyage, de voyage dans le temps. Et il en va de même dans le \emph{Lexikon-Roman}, qui se passe sur le Danube, qui passe juste ici [il montre le Danube par la fenêtre], et qui décrit un voyage de Vienne, plus exactement de Nußdorf, jusqu'à Spitz, dans la Wachau. C'est disons la colonne vertébrale, le fil rouge du roman, ce voyage sur un bateau de croisière sur le Danube. Pendant ce voyage, le protagoniste se perd en rêveries, a toutes sortes d'associations d'idées, il lui revient beaucoup de moments passés à l'esprit, il voit la nature, il voit de jeunes femmes, il s'enivre de la beauté de ces jeunes femmes, laisse aller son imagination. Et on ne peut pas décrire une telle expérience de manière linéaire. Il y a des structures qui jaillissent, que l'on poursuit à l'envi.

\textbf{RV ---} Comment devrait-on écouter la \emph{Lexikon-Sonate} ? Ou du moins comment recommanderiez-vous de l'écouter ?

\textbf{KHZ ---} Cela dépend en grande partie du contexte dans lequel elle est jouée. Il y a deux modes de jeu différents. Le premier est le mode \guill{installation}, dans lequel on lance le programme, le morceau commence à se composer, et joue. C'est un mode qui peut fonctionner dans une exposition, ou comme une sorte d'installation sonore, dans laquelle les visiteurs peuvent entrer à tout moment, et repartir de même, parce qu'il n'y a pas de début ni de fin. C'est quelque chose que je fais volontiers, s'il y a un piano contrôlable par ordinateur, un Bösendorfer CEUS ou un Yamaha Disklavier par exemple. Le programme tourne pendant quelques semaines, et l'on peut entrer dans la pièce et s'y arrêter aussi longtemps qu'on le souhaite. C'est une possibilité. Il y a toutefois un deuxième mode, le mode de concert, que j'ai développé plus tard que le premier. Cela revient à forcer l'hermétique du programme, en remplaçant les automatismes par une interaction en direct. Je peux ainsi jouer de véritables pièces, ce que je fais parfois. Je fais tout simplement une improvisation avec la \emph{Lexikon-Sonate}. Cela pourrait aussi bien être un morceau pour piano très virtuose en provenance de la \emph{New Complexity}. Dans ces cas-là, j'ai en tête une sorte de parcours, ou l'organisation d'une évolution formelle, et le résultat peut tout-à-fait être entendu comme un morceau pour piano traditionnel.

\textbf{RV ---} Et pour ceux qui téléchargent le programme ?

\textbf{KHZ ---} Ils font comme bon leur semble, c'est un cadeau ! On peut, de la même manière, lancer le programme, et le laisser jouer, et profiter de la musique qui est produite, et on peut aussi expérimenter. Dans le \emph{shareware}, dans la version du publique, il est possible d'appeler des structures spécifiques en appuyant sur les touches du clavier, et de connecter un contrôleur MIDI. L'utilisateur a alors à disposition le même dispositif que dans la version de concert que j'utilise. C'est en quelque sorte une manière de réfléchir comme en privé avec cette structure compliquée, comme la contemplation d'une image.

\textbf{RV ---} Quelles ont été les étapes importantes dans l'évolution sur vingt ans de cette pièce ?

\textbf{KHZ ---} Elle était relativement tôt dans le même état que ce qu'elle est aujourd'hui, quoique seulement dans ce mode \guill{installation}. Le progrès le plus important a été l'ouverture des structures, de l'hermétisme, en installant des contrôles directs, qui permettent d'en faire aussi un instrument. C'était le pas le plus important. Ensuite il y a eu de petites adaptations, l'optimisation des algorithmes, mais au fond le programme s'est fait dans les premières années, et n'a été que légèrement modifié par la suite.

\textbf{RV ---} Et pourquoi avez-vous retiré la sortie MIDI du programme ?

\textbf{KHZ ---} Je vais vous dire pourquoi. J'ai découvert que des personnes l'ont utilisé pour générer des partitions. Comme si c'était un programme générateur de partitions. Et ils ont ainsi créé des morceaux qu'ils ont vendu comme leurs propres morceaux. J'ai trouvé que c'était une idée vraiment stupide, parce qu'elle va complètement à l'encontre de ce concept d'une musique éphémère, que l'on ne peut pas fixé. Je l'ai fait moi-même, j'ai installé le même programme de notation, et ai obtenu une partition qui avait une allure incroyable. Mais cela contredit l'idée même de la musique. Ce n'est pas une pièce qu'il faut reproduire, rejouer sur les touches du piano, parce que ce n'est pas une pièce pour pianiste, mais une pièce pour ordinateur. Comme ces gens ne l'avaient pas compris et utilisaient la pièce comme un morceau de papier-toilette, j'ai désactivé cette fonctionnalité. Je l'ai désactivée en connaissance de cause, en sachant que l'inconvénient est que cette version publique ne peut désormais plus fonctionner avec un piano MIDI. Et lorsque quelqu'un qui veut faire des représentations avec piano me contacte, s'il m'assure de manière convaincante qu'il n'en fera que cet usage, je lui envoie la version complète. Mais sur la version en ligne, que n'importe qui peut télécharger, ce n'est pas possible. Il se trouve qu'entre-temps QuickTime a mis au point de bien meilleurs sons de piano, de meilleure qualité. Au début cela avait vraiment l'air de sons de \emph{Playstation}, à la \emph{Super Mario}. Horrible. Mais il y a maintenant de meilleurs sons. Le résultat n'est toujours pas spécialement bluffant, mais au moins cela permet de rendre compte de la pièce.

\textbf{RV ---} Est-ce que les partitions ainsi générées étaient jouables par un pianiste humain ?

\textbf{KHZ ---} Pas vraiment, mais on peut toujours s'arranger. On peut par exemple utiliser le programme pour générer des structures d'accords\dots~Mais ce n'est pas l'idée, et puis il y a toujours le problème de la paternité de l'œuvre. Pour moi, quand les gens font ça, ils devraient au moins citer mon nom. Pas nécessairement me verser des droits d'auteur, mais au moins une forme de référence. S'il n'y en a pas, alors je ne suis tout simplement pas d'accord pour partager mes travaux.

\textbf{RV ---} Est-ce que cela n'est pas gênant que la pièce soit complexe, mais ne se répète jamais ? Lorsqu'on écoute une musique complexe, n'est-on pas souvent tenté de la rejouer, afin de la comprendre mieux ou d'en profiter mieux ?

\textbf{KHZ ---} C'est justement mon intention ! Dans la vie non plus il n'est pas possible de revenir en arrière, de dire \guill{j'aimerais revivre ceci ou cela}. Si vous avez un accident de voiture, vous ne pouvez pas vous dire \guill{je recommence mon trajet à travers la ville, et cette fois je fais bien attention, et je n'aurai pas d'accident}. Il se produit des événements qu'on ne peut pas recommencer. Cela correspond à la réalité de notre vie, et c'est ce que représente aussi la \emph{Lexikon-Sonate}. Cela éveille chez l'auditeur une nouvelle forme d'attention, parce qu'il sait que cela ne se répétera pas, qu'il doit être présent à tout moment. Et quand un bon passage est produit, c'est un cadeau, comme la rencontre d'une personne par le hasard, une possibilité qu'on a une seule fois, dont il faut se réjouir au lieu d'imaginer que l'on peut tout reproduire.

\textbf{RV ---} Pourquoi y a-t-il une différence de complexité entre les différents modules ? Par exemple, pourquoi est-ce que le module \module{Esprit} est beaucoup plus complexe que le module \module{Glissandi} ?

\textbf{KHZ ---} Comme je l'ai dit, parce que la \emph{Lexikon-Sonate} essaie en quelque sorte de synthétiser des \emph{topo\"\i} musicaux. Il y en a beaucoup qui sont simples, et d'autres qui sont très compliqués. Et même quelque chose de très simple comme une basse d'Alberti chez Mozart nécessite un pendant compliqué, une mélodie élaborée, et de la combinaison des deux il émerge quelque chose d'intéressant. C'est aussi comme ça que fonctionne la \emph{Lexikon-Sonate}.

\textbf{RV ---} Et pourquoi la différence entre la complexité totale de trois modules superposés, et la manière très simple de les combiner, simplement en les jouant indépendamment les uns des autres ?

\textbf{KHZ ---} Cela aussi a à voir avec la réalité de notre vie, dans laquelle beaucoup de choses se produisent en même temps, sans lien entre elle, mais qui sont perçues comme un tout. Ce n'est pas seulement une croyance personnelle. Les recherches sont le cerveau ont montré que l'homme (ou plus exactement, le cerveau) est ainsi construit, qu'il essaie de donner du sens à tout ce qu'il reçoit. Le cerveau tente de filtrer à travers le chaos d'informations en provenance des sens dont il est bombardé des connexions pertinentes. C'est aussi un thème présent dans la \emph{Lexikon-Sonate}, le fait que ces trois générateurs soient exécutés en même temps de manière totalement indépendante. Lorsque l'on écoute le programme ont ne s'en rend pas compte, parce qu'on est tout le temps occupé à relier les choses les unes aux autres. Il arrive qu'on puisse percevoir les différents niveaux, parce que les modules qu'on entend peuvent être très différents, et alors on peut suivre facilement : ici les trilles, ici les \emph{glissandi}, ici la mélodie. Et alors, quand les modules se différencient clairement, on peut se dire : \guill{tiens, maintenant j'aimerais entendre plutôt les trilles}, ou la mélodie, ou les \emph{glissandi} ; on peut se concentrer à l'intérieur des perceptions sur les différentes structures, écouter et suivre les différentes couches.

\textbf{RV ---} L'une des raisons du succès de la \emph{Lexikon-Sonate} serait donc le fait que c'est une excellente métaphore de notre vie ?

\textbf{KHZ ---} Oui, je me réjouirais qu'elle soit perçue comme ça. D'une certaine manière, c'est évidemment une pièce hermétique, ou du moins elle l'était au départ. Mais derrière cette hermétique se cachent beaucoup de choses qui sont en rapport avec notre vie actuelle. Et peut-être est-ce, comme vous le dites, ce qui parle aux gens. Ce n'est pas de la musique simple, c'est même de la musique très complexe, mais elle construit quelque chose, que nous vivons peut-être quotidiennement, mais qui est reflété, transformé esthétiquement.

\textbf{RV ---} Alors c'est toujours une expérience mémorable.

\textbf{KHZ ---} Et toujours une surprise.


\subsection{Performance \emph{Oh Nacht, oh Schweigen, oh todtenstiller Lärm!}}

\textbf{RV ---} Hier vous portiez des vêtements classiques, mais sur le programme vous avez un T-shirt (avec un dessin de la spirale de Fibonacci). Est-ce que la façon dont vous vous habillez est importante pour vous lorsque vous faites des concerts ?

\textbf{KHZ ---} Nous y avions effectivement réfléchi. Agnes m'avait demandé quelques jours avant le concert comment nous devions nous habiller. J'ai répondu que je viendrais tout en noir, parce qu'il est question de la nuit. Et elle a répondu que si je venais tout en noir, elle viendrait tout en blanc, et que cela représenterait les polarités.

\textbf{RV ---} L'aspect visuel est donc important dans votre pratique artistique ?

\textbf{KHZ ---} Absolument. Pour moi, c'est une partie de la performance. Quand on est sur scène, on n'est plus en privé. Je ne jouerais pas avec mes vêtements de tous les jours. Ça m'est arrivé avant, mais je suis arrivé à la conclusion que ça n'allait pas. Quand je suis au studio ou en privé, je m'habille comme je veux, mais de même quand par exemple j'enseigne, à l'université, j'ai toujours une veste, je suis toujours en noir, toujours élégant. C'est le principal : ce n'est pas la vie de tous les jours. En cours, à l'université, sur scène, je suis toujours correctement habillé. Je ne mets jamais de cravate, rarement des chemises, mais j'ai toujours une veste et des jeans noirs, et un T-shirt noir. Ce sont des vêtements standard, qui toujours les mêmes.

\textbf{RV ---} Est-ce que cela ne vous a pas dérangé d'avoir autour de vous pendant la performance des gens qui se déplaçaient et faisaient des photos avec des appareils bruyants ?

\textbf{KHZ ---} Non, c'étaient mes fils, qui sont très précautionneux et très sensible, ils font ça à merveille. Cela ne m'a pas dérangé. Il y a des cameramen qui me gênent énormément, parce qu'ils n'ont pas de sensibilité, et qu'ils ne font pas attention. Mais mes deux fils hier l'ont parfaitement fait. Est-ce que c'était dérangeant pour le public ?

\textbf{RV ---} Non, moi ça ne m'a pas dérangé, mais parfois j'étais étonné de les voir passer si près de vous et que cela ne vous ennuie pas.

\textbf{KHZ ---} En fait je m'en suis à peine rendu compte. Et puis, comme ce sont mes fils, je suis à l'aise quand ils sont là, on pourrait presque dire que c'est une joie. Et ils font souvent des prises de vue pendant mes performances, et eux-mêmes sont musiciens, ils font de la musique électronique, et donc ils savent ce que c'est que d'être sur scène et de jouer ou chanter.

\textbf{RV ---} Comment les textes de la performance ont-ils été choisis ?

\textbf{KHZ ---} Dans notre projet \emph{out of the blue}, c'est Agnes qui est responsable des textes, et moi des sons. Elle a donc travaillé intensivement sur Nietzsche. Tout ce projet était à l’origine une invitation à un projet de recherche international avec différents scientifiques et musiciens, auquel nous avons participé l'année dernière à Vienne. Hier j'ai dit un mot à ce sujet à la fin de la performance. Il y avait deux personnes dans le public qui m'y avaient invité, ils voulaient faire quelque chose sur Nietzsche et m'ont demandé ma contribution. J'ai répondu que je participerais volontiers, mais seulement avec Agnes ! Que nous ferions une performance avec texte et musique. C'est comme ça qu'Agnes en est venue à Nietzsche, dont elle ne connaissait pas spécialement les travaux auparavant. Mais elle s'y est beaucoup intéressée, a beaucoup lu, toutes sortes de sources, des poèmes, des textes philosophiques, des extraits de son journal, des lettres, et a fait un montage avec le tout. Elle en avait un plein carton ! Et elle en a fait trois parties, pour trois pièces différentes, et elle a collé ces textes sur des feuilles, formant une sorte de partition textuelle. Mais l'idée n'était pas de les lire du début à la fin, plutôt d'avoir des \guill{patches} avec lesquels improviser.

\textbf{RV ---} Et j'imagine qu'il est plus facile d'improviser à partir d'un matériau textuel qu'on connaît bien.

\textbf{KHZ ---} Oui, c'est toujours extra. Agnes est au départ une chanteuse de jazz, et elle fait aussi beaucoup d'improvisations libres sans texte, juste avec la voix, tout ce que la voix peut faire. Avec le texte, on a en plus quelque chose qui donne de la matière en permanence, qui inspire. Et cela fonctionne à merveille !

\textbf{RV ---} Un peu comme vous avez votre \maze, non ?

\textbf{KHZ ---} Exactement. Même si moi je travaille à la base avec des échantillons sonores, qui sont transformés et assemblés ; ce qu'elle fait avec les textes et la voix est du même ordre. Oui, en fait c'est très semblable.

\textbf{RV ---} J'ai entendu des références à \emph{Also sprach Zarathustra} de Richard Strauß pendant la performance. Est-ce que vous les avez générées complètement en temps réel, ou alors vous aviez préparé votre instrument, par exemple codé des \emph{Strukturgeneratoren}, pour pouvoir lancer facilement ces références ?

\textbf{KHZ ---} J'ai sélectionné des morceaux de l'introduction, certains passages des deux premières minutes, et travaillé avec cela. Mais j'ai aussi préparé une sorte de structure, comme une scène ou un \emph{preset}. Quand la situation se présentait, je pouvais ainsi appeler facilement les bons \emph{Structurgeneratoren} du \maze, avec les bons \emph{samples}, les bons paramètres pour les algorithmes (en termes de transpositions par exemple, d'harmonie, il faut régler les transpositions pour que les nouvelles fonctions que j'appelle soient harmoniquement compatibles avec ce qui s'est fait avant). Quand j'utilise la synthèse granulaire, c'est pareil, je dois définir dans quelle domaine elle s'exprime, quels paramètres elle contrôle, quelle est la taille des \guill{grains}, etc. Et donc tout ceci je le définis à l'avance, comme un \emph{preset}, et grâce à ça je sais que je n'aurai pas besoin de réfléchir à des concepts abstraits comme la taille des grains. Si j'appelle un générateur, je sais exactement ce qu'il va réaliser, quelque chose que j'ai quasiment composé au préalable, et à partir de quoi je peux ensuite improviser.

\textbf{RV ---} Est-ce que vous faites souvent un discours à la fin du concert ? Et est-ce que vous restez souvent dans la salle pour discuter avec le public ?

\textbf{KHZ ---} Cela dépend du contexte. C'est la première fois que je parle \emph{après} le concert. C'est simplement parce que la pièce part vraiment du rien. Ce n'était pas possible de venir au début, de repartir et de changer les rôles, de dire \guill{j'organise le concert, je vous explique un peu la pièce}, et puis revenir \guill{maintenant je suis le musicien}. C'était trop compliqué. Donc hier j'ai fait le discours après. D'habitude, c'est Agnes qui salue le public au début, à la manière d'une actrice, et qui explique aux gens ce que nous allons faire. Souvent le public ignore ce que nous faisons, imagine par exemple que nous jouons des morceaux existants que nous aurions répétés, que nous suivons un programme. Agnes explique que nous travaillons à partir de texte, que nous faisons une improvisation libre, et que nous ignorons où cela va nous mener. Et là souvent ils sont très déconcertés, ou enthousiastes, et puis une fois que ça commence, ils voient comment cela fonctionne, et le lien avec le public est ainsi créé.

\subsection{La composition en temps réel et la composition algorithmique}

\textbf{RV ---} Avez-vous déjà eu des problèmes avec la complexité de vos algorithmes ? Par exemple, avez-vous déjà voulu implémenter des algorithmes qui en réalité étaient trop complexes pour pouvoir être exécutés en temps réel ?

\textbf{KHZ ---} Jamais. Quand on se lance dans l'approche du temps réel, il n'y a que certaines choses qui fonctionnent. Par exemple des choses comme le \emph{backtracking}, ça ne fonctionne pas en temps réel. On ne peut pas générer d'abord une structure, puis la comparer, l'optimiser, ça n'est pas possible. Il faut emprunter un tout autre chemin.

\textbf{RV ---} J'ai l'impression que cela est aussi lié à MAX. Qu'on n'est par exemple pas amené à imaginer les algorithmes en termes de boucles lorsqu'on utilise MAX.

\textbf{KHZ ---} Oui, oui. En fait la structure de la boucle, comme on en trouve dans des langages comme le C, est traitée autrement. C'est une approche complètement différente. Et c'est peut-être une très bonne chose, de quitter cette représentation avec des boucles. \guill{\emph{for}}, \guill{\emph{if}}, \guill{\emph{while}}, etc. Il faut trouver d'autres moyens. Mais comme tout fonctionne en temps réel, et que le temps passe, c'est le temps qui est le facteur limitant. C'est lui qui conditionne comment on joue ou comment on fait de la musique. Ce n'est pas le cas quand on compose. Quand on compose, le fonctionnement de base est différent : on peut à chaque instant revenir en arrière, fabriquer des relations ou les améliorer. Mais quand on improvise, en direct, on est obligé de travailler avec le temps. Rien de ce qui se passe ne peut être répté --- bien sûr on peut essayer, de revenir en arrière, de faire quelque chose de similaire, mais toujours en naviguant dans l'écoulement du temps.

\textbf{RV ---} Lorsque vous composez \guill{hors du temps}, comment utilisez-vous l'ordinateur ? Est-ce que vous êtes plutôt occupé à implémenter des idées musicales, ou alors à expérimenter avec l'ordinateur à la recherche de telles idées ?

\textbf{KHZ ---} C'est très variable. Il y a des pièces que je fais entièrement sans ordinateur. La plupart de mes pièces instrumentales, je les compose sans ordinateur. Je travaille beaucoup avec le papier et le crayon, et beaucoup avec les instruments. L'ordinateur sert principalement dans les pièces avec de l'électronique, comme une srote de succédané instrumental, ou en tant que niveau instrumental dans le jeu. L'ordinateur est un instrument dans ces cas-là. Dans les années 90, j'avais composé beaucoup de pièces de musique instrumentale avec des algorithmes, mais je ne le fais plus aujourd'hui. Je m'y remettrai peut-être, comme avec cet Opusmodus dont je vous ai parlé, cela m'intéresserait, mais ces dernières années j'ai écrit mes œuvres instrumentales absolument sans ordinateur. Je me suis rendu compte que dans certains domaines j'étais bien meilleur que l'ordinateur, par exemple pour l'harmonie, les hauteurs de notes. J'ai plus d'expérience, de meilleures représentations, de meilleures compétences que l'ordinateur. Il serait plus pénible de programmer que de le faire moi-même, et je n'en ai ni le temps ni l'envie. Avec l'ordinateur, cela me prendrait sûrement cinq fois plus longtemps.

\textbf{RV ---} J'ai lu quelque part, ce qui m'a un peu étonné, que vous parliez surtout de la structure de vos pièces parce que vous trouviez que parler des hauteurs de notes était trop personnel\dots~Qu'est-ce que ça veut dire ?

\textbf{KHZ ---} Oui, c'est juste ! C'est en quelque sorte un savoir-faire secret que je manie, qu'on ne peut pas théoriser. Je me souviens que j'avais écrit une pièce pour piano jouet et ensemble, \emph{under wood}, et qu'une musicologue a écrit un commentaire sur la pièce et en a fait une analyse, mais seulement de la structure harmonique. Et elle a mis au jour des choses incroyablement intéressantes, exactement la façon dont je pense ! Elle a dessiné un tableau d'intervalles, et m'a même demandé si c'était bien, et j'ai répondu : \guill{oui, absolument !} [il rit]. Mais ça n'a pas tellement d'importance, c'est de l'artisanat. J'en ai besoin pour mettre en place les espaces sonores, pour que tout sonne ensemble de manière cohérente. Mais ça n'a pas une importance considérable, ce qui compte c'est plutôt la façon dont les structures et les sons qui naissent et qui sont intégrés à cette harmonie. C'est intéressant pour la théorie musicale --- la discipline théorique, qui est très orientée vers les hauteurs de notes. Dans la musique que j'écris l'élément central est plus le son ou d'autres choses, ce qui n'est en fait jamais pris en compte en analyse. L'harmonie est presque secondaire, c'est à l'arrière-plan. C'était aussi sa conclusion.

\textbf{RV ---} Et comment pourrait-on donc faire une analyse d'une telle pièce, par exemple de la \emph{Lexikon-Sonate} qui est plus un processus qu'une pièce finie ? Est-ce qu'une analyse harmonique serait possible ?

\textbf{KHZ ---} Harmonique, non, parce que l'harmonie est très complexe. Elle est organisée différemment dans chaque module. Par exemple \module{Esprit} fonctionne avec une sorte de mouvement brownien, avec certaines interdictions de répétition d'intervalles et de hauteurs de notes. Mais c'est généré à partir de principes aléatoires. Alors que d'autres générateurs, par exemple \module{Chords} ou \module{BrownChords}, reposent sur des structures d'intervalles. L'harmonie est donc à un niveau supérieur : les structures sont aussi déterminées par le hasard, mais il y a un système de règles contraignant, qui interdit certaines combinaisons d'intervalles, exclue certains accords. Il en résulte des sons bien spécifiques. Par exemple, il ne peut pas y avoir d'accords majeurs et mineurs en même temps. Il y a des accords de trois notes, mais tout ce qui sonne tonal est stratégiquement mis entre parenthèses.

\textbf{RV ---} Alors que faut-il faire, une analyse des algorithmes ?

\textbf{KHZ ---} Sûrement, oui. Non pas du résultat, de l'\emph{output}, mais de la construction des algorithmes.

\textbf{RV ---} D'après vous, quelle est l'importance de savoir comment un morceau est construit pour l'apprécier ?

\textbf{KHZ ---} C'est une excellente question. Chez les poste-modernes, il y a le concept du double codage [\emph{Doppelkodierung}]. Cela signifie qu'une œuvre d'art n'est pas univoque, mais parle plusieurs langues en même temps, et que l'auditeur peut entendre des choses très diverses. Il y a le mode de l'écoute naïve, quand on n'a aucune information, et qu'on laisse la pièce faire effet, que cela peut initier quelque chose, ou non. Il y a aussi l'écoute du spécialiste, quand on est informé et qu'on envisage déjà les choses selon une certaine direction, qu'on accorde une attention différente en lien avec ces informations préalables. Et entre deux il y a toutes sortes de possibilités. Je crois que les deux sont possibles, et qu'il y a plusieurs sortes d'écoutes entre celle du spécialiste et la naïve, l'innocente. Mais je dis souvent que les gens devraient pouvoir venir à la pièce sans idée préconçue. Parfois avoir un contexte explicatif est important. Par exemple, ici [il montre un instrument], cet instrument est une pipa. C'est un instrument chinois avec une très longue histoire. Il est joué principalement par les femmes, et est très virtuose. Là-bas [idem] il y a un oud arabe. Ce qui est intéressant, c'est que l'oud, qui est un instrument vieux de mille ans, a un jour réussi à voyager jusqu'à la Chine, et les chinois en ont construit d'autres, et c'est devenu la pipa. Comme ils utilisent une échelle différente, ils ont ajouté des languettes de bois qui fixent les hauteurs de notes (l'oud n'a pas de languettes parce qu'il y a le système \emph{maqâm} avec des micro-tons). J'ai écrit une pièce pour pipa et musique électronique \emph{live}, et ce n'était pas simplement une pièce avec un instrument lambda, mais j'ai travaillé volontairement avec l'histoire de cet instrument. J'ai trouvé un poème d'un poète chinois très célèbre du \VIIIe~siècle, qui est au sujet de la pipa. Il est plein d'images de nature, et décrit une soirée de fin d'été, où la lune se lève, où le fleuve coule, et il y a des amis qui font la fête ensemble, qui se disent au revoir, et soudain ils entendent le son d'une pipa. Il remontent jusqu'à la source du son et voient une femme qui joue de la pipa et qui est triste, désespérée. Une sorte de poème d'amour. Et moi qui voulais que l'histoire de l'instrument joue un rôle d'une manière ou d'une autre, j'ai composé la pièce de manière à ce que le poème apparaisse. J'ai demandé à la joueuse de pipa, qui est chinoise, de dire le poème, j'ai gardé quelques vers, et j'ai fait en sorte qu'au cours du morceau l'instrument se mette soudain à parler. J'ai utilisé de la convolution, à certains moments il y a ce son électronique qui est déclenché par la pipa, comme une voix humaine. C'est donc un morceau qui travaille avec et fait référence à une certaine forme de tradition, entre l'Asie et l'Europe --- une différence colossale dans la manière de penser. Et j'ai essayé de créer un lien malgré tout. Nous avons joué ce morceau plusieurs fois à Taïwan, et les gens l'ont beaucoup apprécié, parce qu'ils ont tout compris. Mais quand le joue en Autriche, on entend bien que la pipa commence à parler, mais dans une langue totalement incompréhensible. Pour les Chinois c'était extra, parce qu'en plus ils connaissaient le poème. Dans toutes les cultures, il y a des poèmes que tout enfant connaît. Voilà, je me suis un peu éloigné du sujet, mais pour ce qui est de l'information de l'auditeur : parfois c'est important, parfois je fais des références explicites, parfois j'essaie aussi de l'éviter.

\textbf{RV ---} Pourquoi citez-vous peu Xenakis ?

\textbf{KHZ ---} Oh, c'est une très bonne question, parce qu'il y a plein de choses chez lui qui m'intéressent, et puis j'ai lu \emph{Formalized music}\dots~Je ne sais pas ! Je crois que pour moi ce n'est pas une référence absolue, mais ça vient de considérations extra-musicales. C'est un travail que j'apprécie beaucoup mais qui n'est pas vraiment proche de ce que je fais ; mon approche est plutôt marquée par le sérialisme, qu'il a violemment rejeté. Mais au fond ces deux approches ont fini par se réconcilier, se lier.

*********************************************************************

\textbf{RV ---} Quel est votre avis sur le logiciel libre ?

\textbf{KHZ ---} Sur le principe, je trouve que c'est une idée grandiose. Je souffre de ce que je travaille peu avec des logiciels graphiques, qui ne fonctionnent que dans certains environnements, et si je change d'ordinateur ça ne fonctionne plus, je dois acheter un nouveau code\dots~C'est vraiment pénible. Et pourtant je suis parfaitement prêt à dépenser de l'argent pour des logiciels. Reaper, par exemple. On peut payer ou non, et évidemment j'ai payé, parce que je trouve que c'est un bon programme, sur lequel je sais que je peux me reposer, qui fonctionne, qui est simple, et je n'ai besoin de m'enregistrer à chaque fois pour le faire fonctionner. Ça a aussi l'avantage que, quand ils sont bien faits, on peut vraiment compter sur eux, ils sont très stables. Et puis il y a la possibilité de contribuer à les développer.

\textbf{RV ---} Et pourquoi dans ce cas utilisez-vous un ordinateur Apple ?

\textbf{KHZ ---} Parce que c'est avec un Mac que je suis le plus productif. Je ne suis pas un partisan d'Apple, dans le sens où il n'est pas nécessaire que ce soit un Apple. Mais quand je travaille avec des étudiants qui ont un PC plutôt qu'un Mac, c'est fastidieux de faire des choses similaires, sauf quand les gens sont vraiment très doués et savent travailler avec leur système d'exploitation à un niveau plus profond. Mais pour un utilisateur moyen, dans le domaine multimédia, un ordinateur Windows sera toujours inférieur pour un habitué d'Apple. C'est aussi en rapport avec le système d'exploitation, avec l'interface graphique, avec toute la philosophie qu'il y a derrière. C'est tout simplement mieux traité par Apple. Le problème, c'est qu'Apple devient de plkus en plus un produit de consommation, et beaucoup de choses qui auparavant étaient prévues dès l'installation ne sont plus si simples aujourd'hui. Je me souviens que j'avais configuré mon précédent Apple exactement comme j'en avais besoin, j'étais beaucoup allé chercher dans le système d'exploitation, et je l'avais préparé pour qu'il m'aille vraiment comme un gant. Aujourd'hui c'est plus compliqué, et ça fonctionne moins bien, donc on ne fait plus ce genre de choses, et je dois de plus en plus m'arranger avec ce qu'il y a. Et puis tout à coup de nouvelles choses sont installées, comme cette mise à jour entre iPhone et iPad, iCloud. Tout devient plus \emph{user-friendly}, mais pour l'utilisateur expérimenté c'est aussi parfois plus pénible. C'est la raison pour laquelle je ne travaille pas avec les systèmes d'exploitation les plus récents. J'en suis encore à la version 9 ou 10, parce que c'est aussi plus stable en termes de compatibilité auprès de mes collègues dans le monde multimédia. J'ai un ordinateur vieux de quatre ans, dont j'ai remplacé le disque dur par une SSD, et depuis les calculs sont beaucoup plus rapides, et tout fonctionne ! Alors, quand on me conseille de dépenser 2000 euros pour un nouvel ordinateur, je réponds : \guill{à quoi bon ?}. Je n'en ai pas besoin, ce que j'ai fonctionne parfaitement, j'en suis totalement satisfait. C'est une question de productivité, j'ai fait l'expérience qu'on se débrouille mieux avec les systèmes d'exploitation d'Apple.

\textbf{RV ---} Et sur votre ordinateur, vous avez caché la pomme d'Apple par un autocollant, avec la mention \guill{E(ART)H}.

\textbf{KHZ ---} Oui, je l'ai collé par-dessus. Je me dis que si je fais de la publicité pour l'entreprise Apple, ils doivent me donner quelque chose en retour, au moins un ordinateur. Ils ne le font pas, donc je ne fais pas de publicité. Et puis ça dérange quand on joue d'avoir toujours cette pomme qui brille. Et je ne veux pas de cette publicité. Cet autocollant que j'utilise vient simplement d'un projet artistique. Je l'ai gardé parce que je le trouvais joli, et pertinent.

\textbf{RV ---} Comment gagnez-vous votre vie ?

\textbf{KHZ ---} J'ai une chaire à l'Académie de musique de Vienne. C'est mon revenu principal. Ça me permet d'avoir une activité libre de compositeur et de musicien. Et puis j'ai aussi eu jusqu'ici une activité de conservateur au musée, d'organiser les concerts. Et j'ai aussi été webmaster du musée. Ce travail ici prend malheureusement fin : j'avais trois activités, et je n'en ai plus que deux.

\textbf{RV ---} Et faites-vous souvent des concerts gratuits, comme hier ?

\textbf{KHZ ---} Les concerts au Essl Museum sont toujours gratuits. Et quand je joue ailleurs, cela dépend. Je ne suis pas l'organisateur, donc ça dépend de l'organisateur. Ça peut être un événement parfaitement normal où les gens paient quinze euros, ou bien gratuit, si c'est dans un autre contexte.

\textbf{RV ---} Avez-vous une opinion sur le problème du téléchargement illégal ?

\textbf{KHZ ---} Ma position personnelle est de payer pour les choses. Quand je veux avoir un morceau de musique qui me plaît, ou quand j'ai besoin d'avoir un certain enregistrement pour un cours, je l'achète, parce que je pense que c'est important pour les auteurs, qu'il faut s'en acquitter auprès d'eux. Comme vous connaissez ma page web, vous savez que je partage beaucoup de choses gratuitement. Quand quelqu'un veut une de mes partitions, il ne doit pas l'acheter, il peut la télécharger gratuitement. À quelques exceptions près, il en va de même pour les programmes. Il y a deux \emph{sharewares} pour lesquels on doit acheter un code d'enregistrement, mais le reste est gratuit. Je le fais sciemment, parce que c'est un moyen de diffuser la pièce plus facilement. Je sais à quel point cela peut être difficile. Si on veut une partition de Fausto Romitelli, un compositeur italien, il faut écrire une lettre à Ricordi [Casa Ricordi, maison d'édition musicale], et attendre la réponse pendant deux mois, et au bout du compte il faut six mois pour que le document arrive, avec en plus la possibilité qu'il soit refusé. J'avais moi-même une maison d'édition, en Allemagne, qui fonctionnait comme ça. Quand des gens voulaient un aperçu d'une de mes partitions, la maison d'édition le leur envoyait, et ça allait aux clients, s'ils voulaient jouer le morceau, ils achetaient la partition. Mais la maison d'édition, mon ancienne maison d'édition désormais, a fait en sorte depuis que ceux qui voulaient voir une partition de moi devaient immédiatement payer pour cela. Et les représentations de ces œuvres ont drastiquement diminué. Alors j'ai acheté ma liberté : j'ai rompu le contrat avec la maison d'édition, j'ai dû payer une certaine somme, mais j'ai récupéré tout le matériau qu'ils avaient, cinquante kilos de papier. Et maintenant je suis libre, j'ai mes propres affaires, et j'ai tout scanné. Quand on veut jouer mes travaux, on peut les télécharger gratuitement. C'est une décision que j'ai prise. J'ai constaté que ma musique se diffusait bien mieux ainsi, et je ne gagne pas d'argent en vendant du papier, mais sur les droits d'auteurs des représentations. Quand c'est un concert officiel, je reçois les taxes prélevées par la SACEM, ou la GEMA, ou l'AKM. Et quand je fixe le prix d'une partition à quinze Euros, mais qu'elle doit être imprimée, qu'il faut aller au \emph{copyshop} pour cela, qu'il faut la faire relier, la mettre dans une enveloppe, faire une facture et que tout cela doit être déclaré au fisc et que l’État autrichien perçoit 50\%~des recettes\dots~Bref, ça ne me rapporte rien. On n'en vit pas. J'ai donc décrété que tout été gratuit, mais aussi toutes les œuvres sont enregistrées, à la SACEM, etc., et je reçois des droits d'auteur quand elles sont jouées.

\textbf{RV ---} Et maintenant votre opinion sur la protection des données privées sur Internet ?

\textbf{KHZ ---} Ce que je vois, c'est que lorsqu'on profite des avantages de Google ou de Facebook, c'est aussi un peu comme si on vendait son âme. Et il faut décider, soit d'accepter ça, soit d'y renoncer. Pour moi, ce sont les avantages qui dominent. Il se trouve que je publie beaucoup de choses. Mon Facebook n'est pas un compte privé, où j'écris des choses privées ou des histoires de famille, ce sont toujours des publications artistiques, en lien avec mon travail artistique. Et je suis heureux que cela soit diffusé. Mais je sais aussi que chaque recherche que je fais sur Google est enregistrée. Tout ce qu'on peut faire, c'est refuser de les utiliser. Je connais même des gens qui ne veulent pas d'ordinateur, pas de téléphone portable\dots~Ils vont ce qu'ils veulent, mais les avantages sont considérables, et moi j'en ai besoin, même si ça revient à \guill{vendre mon âme au diable}. Pour l'instant ça ne m'a pas posé de véritable problème. Mais je suis prudent, par exemple par rapport à ce que je poste. Je réfléchis scrupuleusement à ce que je fais, à ce que j'écris, à comment je le formule, je ne fais rien sur un coup de tête. Je suis conscient que cela va devenir une partie de la bulle Internet, que même si ce n'est pas visible actuellement, cela peut ressurgir n'importe où et à tout moment. Je pense à l'exemple d'un musicien allemand qui est allé aux États-Unis, où il a été arrêté, et les Américains savaient tout de lui en cinq minutes, parce qu'ils avaient épluché son tout ses réseaux sociaux avec des algorithmes spécialisés, et ils savaient tout de suite avec qui il avait été en contact, avec qui il avait téléphoné et combien de temps, etc. C'est un fait, auquel je ne peux rien. Je suis aussi très prudent avec le \emph{cloud}, que j'utilise, mais pas complètement.

\textbf{RV ---} Est-ce que vous vous intéressez aux progrès de l'intelligence artificielle ?

\textbf{KHZ ---} Dans le domaine artistique, pas du tout. Je trouve que c'est intéressant, je lis ce qui s'écrit là-dessus dans les magazines spécialisés, mais en musique ça ne m'intéresse pas. Et je n'ai jamais trouvé l'intelligence artificielle intéressante artistiquement. Parce qu'alors je me demande bien quel est mon rôle en tant qu'artiste\dots~Par exemple, la \emph{Lexikon-Sonate} n'a rien à voir avec l'intelligence artificielle, elle est absolument bête ! Et pourtant elle produit, par là même, quelque chose de beau, parce qu'en tant que créateur du programme j'ai réalisé des idées que j'avais. Même si c'est soit l'ordinateur qui exécute ce que, d'une certaine manière, j'ai proposé, il n'invente rien.

\textbf{KHZ ---} Est-ce que vous vous intéressez aux jeux vidéos, ou plus précisément à la musique de jeux vidéos, qui est une sorte différente de musique algorithmique ?

\textbf{RV ---} C'est un sujet intéressant, mais je n'ai jamais rien fait dans ce domaine. Donc, difficile à dire. J'ai déjà été contacté par des gens qui disaient que mes outils conviendraient merveilleusement à ce genre de choses. Mais il n'y a jamais eu de situation concrète pour réaliser ce genre d'idées. D'une manière générale, je pense que cela m'intéresserait. Aussi, je ne suis pas du tout un amateur de jeux vidéos, même si je me rends bien compte à quel point ces mondes artistiques sont devenus extraordinaires, c'est tout simplement fascinant. Et puis cette interaction, le fait qu'on ne regarde pas seulement un film, mais qu'on produise le film soi-même. Mais je n'ai absolument aucune ambition dans le jeu vidéo. Sur mon premier ordinateur, cet Atari, il y avait seulement deux programmes au départ : BASIC et une fiction interactive. J'avais joué avec cette fiction interactive, et je suis devenu accro\dots~et puis j'ai tout arrêté en me disant que je ne ferais plus jamais une chose pareille, que le danger de s'y perdre était trop grand. Et à la place, je me suis mis à inventer des algorithmes pour faire de la musique !

\textbf{RV ---} Est-ce que vous attachez de l'importance à la qualité des enceintes quand vous faites un concert ?

\textbf{KHZ ---} Une importance extrême, je suis même un grand critique des enceintes normales. Je trouve que c'est un instrument abominable. C'est pourquoi hier nous avons utilisé ces enceintes Bose, des L1. Je crois que c'était la bonne chose à faire, pour ce genre de musique, parce qu'elles diffusent le son très largement, pas comme ces cubes noirs qui diffusent le son comme des coins. Les L1 forment une surface. Et ont une bonne résolution. J'ai trouvé des choses très intéressantes, comme quand je joue dans de petits contextes, de petites salles, où je n'ai pas envie de traîner d'énormes haut-parleurs, et qu'il y a un piano à queue. Dans ces cas-là j'utilise la caisse de résonance du piano comme haut-parleur. J'au une petite enceinte spécialement conçue pour cela, un transducteur. La vibration se transmet à travers la caisse de résonance, qui fait office de membrane de haut-parleur. Mais le son vient du piano. Si on ouvre l'abattant du piano, cela fait une sorte de lentille acoustique, et le son est extraordinaire.

\textbf{RV ---} Oui, je vous ai déjà vu faire ça avec la \emph{Lexikon-Sonate} !

\textbf{KHZ ---} Tout-à-fait. C'est vraiment chouette, parce que ça sonne comme un piano, ça vient du piano, mais je ne joue pas sur les touches, et ce n'est pas non plus un Disklavier. Et je l'ai aussi fait ces dernières années avec des pièces électroniques sans lien avec le piano, et ça fonctionne vraiment bien.

\textbf{RV ---} Y a-t-il une influence du rock audible dans votre musique d'aujourd'hui ?

\textbf{KHZ ---} Seulement à travers cette idée du \guill{\emph{sound}}, dans la plupart de mes pièces électroniques. Ce n'est pas le rythme, pas le \emph{groove}, mais plutôt cette forte orientation vers le travail du son. Quelque part chez moi c'est du rock que ça vient. Ça s'est à nouveau glissé dans ma musique lorsque j'ai commencé à travailler avec de l'électronique, et puis j'ai écris beaucoup de pièce pour guitare électrique, ou ensemble et guitare électrique. Et alors c'était très important de recourir à cette type spécial de \guill{\emph{sound}}, quoique sans tomber dans les clichés de la musique rock.

\textbf{RV ---} Y a-t-il eu des inventions technologiques qui ont été importantes pour votre carrière ?

\textbf{KHZ ---} En tant qu'artiste j'ai toujours voulu avoir une autonomie. Je voulais être indépendant de toute institution. Cela a toujours été important de pouvoir tout faire chez moi. Comme vous le voyez, mon studio tient essentiellement dans mon ordinateur portable. Je n'ai pas besoin de synthétiseur par exemple. Il se trouve qu'il y en a un là [il le montre], par hasard. C'est presque une antiquité, que j'avais achetée il y a longtemps, et qu'en fait je n'utilise pas. La technologie, la technologie digitale, m'a permis d'avoir une existence artistique autonome, indépendante des institutions. Cette autonomie est aussi la raison pour laquelle j'ai toujours refusé de travailler avec les logiciels standard, ProTools Logic et ce genre de choses. Parce que les idées qui sont derrières sont très conservatrices. En tout cas, comme avant quand on travaillait avec des bandes magnétiques ; c'est organisé différemment mais au fond c'est toujours la même idées, avec ces \guill{\emph{timelines}}. L'approche qui m'a toujours intéressé, c'est celle qui ne fonctionne vraiment qu'avec l'ordinateur.

\textbf{RV ---} Et est-ce qu'il y a des inventions ou des progrès que vous attendez ? Que vous espérez ?

\textbf{KHZ ---} Non. Toute cette spéculation sur l'\guill{intelligence} artificielle ne m'intéresse pas, et puis, je ne sais pas\dots~Je vois qu'il se passe beaucoup de choses, mais je n'ai pas envie d'être obligé de suivre toutes les modes. Pendant longtemps j'ai beaucoup expérimenté, et ce qui m'intéresse c'est évidemment comment je peux contrôler les algorithmes. Les gens ont parfois une représentation naïve, surtout ceux qui ne sont pas musiciens, selon laquelle on pourrait sculpter le son ainsi [il fait de grands gestes], comme si on pouvait tailler une sculpture avec des notes. Mais la réalité est que le son est un phénomène. Le son est un phénomène qui n'est pas comparable à un morceau d'argile. Ce n'est pas non plus tridimensionnel, c'est beaucoup plus compliqué. Et voilà, on ne peut pas sculpter des formes, en trois dimensions à travers l'air, avec les mains, et produire de la musique ainsi. Je ne crois pas à ce genre de choses et tout ce que j'ai vu qui prétendait s'en approchait était en fait trivial et ridicule. Pour moi le son c'est une interaction très complexe de beaucoup de paramètres sur plusieurs niveaux, des algorithmes qui se conçoivent mutuellement, et seulement après forment ce qu'on entend, le son comme catégorie supérieure. Cela nous ramène encore au rock, parce que dans le rock le son, \emph{sound}, est un concept qui joue un rôle important. Ces derniers temps j'ai beaucoup utilisé des contrôleurs. J'ai une manière de faire très spécifique, parce que j'utilise des contrôleurs qui existent à un niveau structurel. Des contrôleurs MIDI, que l'on contrôle avec les mains, que l'on sent sous ses doigts. J'ai fait beaucoup d'expériences avec des iPads, ou des tablettes en général. Il y a énormément de possibilités, de logiciels pour créer des contrôleurs. Le problème, c'est que ces contrôleurs ne sont pas tactiles, seulement visuels. Ils donnent certes l'illusion que l'on a affaire à des surfaces graphiques dans lesquelles on peut en quelque sorte plonger ses doigts, mais en définitive il faut regarder l'écran pour trouver le bon endroit. Moi au contraire j'ai un contrôleur vraiment tactile, un contrôleur physique, que je peux jouer en aveugle. Je peux le prendre en main et je sens s'il est haut, pas, à gauche, à droite, c'est quelque chose que je ressens corporellement, que je peux jouer en aveugle, que je n'ai pas besoin de regarder. La différence avec une tablette est considérable, parce que si je joue avec une tablette je dois sans cesse regarder ce que je fais. Et la vue est un sens analytique, ce qui signifie qu'en quelque sorte l'intuition est laissée de côté. Il y a toujours une couche interprétative intermédiaire. Au contraire, ce qui est tactile est beaucoup plus intuitif, beaucoup plus corporel, direct, bref ça relève moins du cortex. Je dois dire qu'il m'arrive d'utiliser des tablettes, parce que c'est pratiques, parce qu'on peut se déplacer, mettre des choses en place dans la pièce sans se presser, parce que tout passe pas les ondes. Mais pour le jeu lui-même je ne suis pas convaincu, et même je trouve ça épouvantable quand les gens regardent leur tablette du début à la fin. C'est contre-intuitif. Peut-être que d'autres savent en tirer meilleur parti que moi, mais chez moi c'est hors de question. Je trouve aussi les contrôleurs gestuels intéressants, comme les \emph{Leap Motions}. Enfin je trouve le concept intéressant, mais de même, le problème est que ça ne fonctionne qu'avec un feed-back visuel. Certes on fait des mouvements dans l'espace, mais si on n'a pas d'écran sur lequel regarder ce que cela change, on est perdu. On se retrouve souvent hors du champ de l'écran et ça ne fonctionne plus. Et je voudrais dans la mesure du possible désactiver cette composante visuelle. C'est souvent un gros problème, et c'est pourquoi moi-même je m'assure toujours que sur l'interface de mes logiciels ne sont toujours allumés que les éléments totalement indispensables, que tout le reste est caché, et que le plus de choses possibles fonctionnent avec des contrôle tactiles qui ont des frontières bien délimitées. Il y a ce qu'on appelle des \guill{contrôleurs continus}, que l'on peut tourner dans n'importe quel sens, mais le problème c'est que je ne peux pas savoir quand j'ai atteint le maximum ou le minimum, il faut que je retourne voir où j'en suis, et lire que j'ai atteint 0 ou 127. Il faut encore faire appel à la vue, et c'est mauvais. Ma philosophie : il faut ressentir ce qu'on fait. Dans la performance d'hier, il y a deux moments où j'ai fait des mouvements en l'air avec ma main ; j'utilise la webcam de mon ordinateur, la vidéo est analysée, et ça me permet de jouer sur trois paramètres à la fois (droite/gauche, haut/bas, avant/arrière). J'utilise une forme spéciale de synthèse granulaire, qui me permet par exemple de me déplacer à l'intérieur d'un échantillon sonore (gauche/droite), la hauteur permet de régler la taille des grains, et la profondeur, leur densité. Ça, je l'avais programmé moi-même, et je me suis rendu compte qu'il fallait d'abord que j'invente des gestes avec lesquels je peux faire de la musique. Ce n'est pas comme si je faisais des mouvements qui sont ensuite traduits en son, c'est même l'inverse. J'avais un système qui fonctionnait d'une certaine manière, et je devais trouver des mouvements pour que le système puisse jouer avec. Ça fonctionne plutôt bien. Cela signifie à nouveau pour moi un contrôle visuel, parce que je dois voir la courbe du signal, et mon mouvement ; je vois comment la fenêtre change et, évidemment, j'entends ce qui se passe. Mais je ne pourrais pas jouer en aveugle, j'ai besoin de regarder l'écran. J'ai un fedd-back visuel, je me vois à l'écran, et je sais où est ma main, ce qui est visible ou non, et en-dessous la courbe du signal, je sais où j'en suis, ta taille de la fenêtre, etc.

\textbf{RV ---} Quelle est votre relation au Essl Museum ?

\textbf{KHZ ---} C'est un musée que mes parents ont fait construire. Mes parents sont collectionneurs d'art depuis des années et ils ont décidé il y a vingt ans de rendre leur collection publique. Ils ont donc fait construire un musée, qui a été en activité pendant six-sept ans. Le problème c'est que l'entreprise de mon père a eu des difficultés financières, à cause de la crise financière, et il a fallu la vendre, or le musée n'était entretenu qu'avec des fonds privé, qui n'existent plus. Et comme nous n'avons pas eu de soutien des collectivités publiques, mon père a dû décider, le cœur lourd, après toutes sortes de combats et de négociations, de fermer le musée. Le bâtiment sera toujours utilisé comme dépôt, toutes les salles au rez-de-chaussée sont des salles d'entrepôt, où les œuvres sont stockées de manière professionnelle. C'était une décision vraiment très difficile. Et les concerts que je fais ici, c'est une rampe de lancement pour les musiciens qui font des concerts de musique expérimentale, qu'on ne pourra pas recréer ailleurs. Enfin, pas dans une salle de concert ou une philharmonie, mais dans des formats plus petits. J'ai beaucoup travaillé avec des gens qui sont aussi un peu expérimentaux, qui tentent de nouvelles choses, et j'avais assemblé une sorte de petite communauté, qui existe toujours.

\textbf{RV ---} Quelle est le lien entre votre projet artistique et celui du musée ?

\textbf{KHZ ---} Quand le musée a été construit, l'architecte a dit dès le début que la musique devrait jouer un rôle important. C'est pourquoi il a construit ce studio. C'est une pièce qui est en fait un cube : six mètre par six mètres par six mètres. En fait il a dit que c'était la structure de base du bâtiment, qu'il devait impérativement y avoir quelque part une pièce de six fois six fois six mètres. Ce n'est pas une pièce qu'on utiliserait pour un bureau. Il a donc ajouté qu'il devait y avoir là un compositeur. Et il a construit comme ça dès le départ, de façon à ce que dans les différentes pièces il y ait des câbles de haut-parleurs. Au début, pendant les dix premières années, j'ai fait beaucoup d'installations sonores ici, avec différents artistes. L'idée était de construire des expositions avec des environnements sonores très particuliers. Cela m'a permis de beaucoup expérimenter. Et j'ai découvert que cette approche générative était très importante. Quand les visiteurs entendent en permanence la même boucle, et c'est encore pire pour ceux qui travaillent dans le musée, comme les gardiens, s'ils entendent pendant trois mois la même boucle toutes les deux minutes, ils vont devenir fous. Et donc il était clair que la musique que nous utilisons ici devait évoluer continuellement, sans jamais cesser d'être intéressante. Ça a été en contact très étroit avec mon propre travail artistique, et j'ai beaucoup appris de ce travail avec le public, de voir comment il réagissait.

%\newpage
%\nocite{*}
%\section{Bibliographie}
%
%\defbibheading{interviews}{\subsubsection*{Interviews}}
%\defbibheading{articles}{\subsubsection*{Articles théoriques}}
%\defbibheading{oeuvres}{\subsubsection*{Œuvres}}
%\defbibheading{secondaires}{\subsection*{Sources secondaires}}
%\setlength{\bibitemsep}{0pt}
%%\setlength{\bibnamesep}{0pt}
%%\setlength{\bibinitsep}{0pt}
%\renewcommand{\bibfont}{\small}
%
%
%\subsection*{Sources primaires}
%\printbibliography[heading=interviews,subtype=interview]
%\printbibliography[heading=articles,subtype=article]
%\printbibliography[heading=oeuvres,subtype=oeuvre]
%
%
%\printbibliography[heading=secondaires,subtype=secondaire]

\end{document}
